\documentclass{article}
\usepackage[a4paper, left=3.5cm, right=3.5cm, top=1cm, bottom=1cm]{geometry}
\usepackage{xcolor,soul}
\usepackage[parfill]{parskip}

\newcommand{\hlc}[2][yellow]{{
    \colorlet{foo}{#1}
    \sethlcolor{foo}\hl{#2}}
}

\pagenumbering{gobble}

\title{Bitcoin Open-Source Implementierung einer P2P Währung\footnote{Übersetzt aus dem Englischen}}
\author{Satoshi Nakamoto}
\date{11.02.2009, 22:27}
\begin{document}
    \maketitle
    Ich habe ein neues Open-Source-P2P-E-Cash-System namens Bitcoin entwickelt.
    Es ist vollständig dezentralisiert, ohne zentralen Server oder Parteien, denen vertraut werden muss,
    da alles auf Kryptografischen Beweisen statt auf Vertrauen basiert.

    [\ldots]

    Das Hauptproblem bei konventionellen Währungen ist das ganze Vertrauen, das nötig ist, damit sie funktionieren.
    Der Zentralbank muss vertraut werden, dass sie die Währung nicht entwertet, aber die Geschichte der
    Fiat-Währungen ist voll von Verstößen gegen dieses Vertrauen.
    Man muss den Banken vertrauen, dass sie unser Geld verwahren und elektronisch übertragen, aber sie verleihen es in
    Wellen von Kreditblasen mit kaum einem Bruchteil an Reserven.
    Wir müssen ihnen unsere Privatsphäre anvertrauen und darauf hoffen, dass sie unsere Konten nicht von
    Betrügern leerräumen lassen.
    Ihre massiven Betriebskosten machen Mikrotransaktionen unmöglich.

    Vor einer Generation hatten Time-Sharing-Computersysteme für mehrere Benutzer ein ähn\-liches Problem.
    Bevor es starke Verschlüsselung gab, mussten sich die Benutzer auf den Passwortschutz verlassen, um ihre Dateien zu
    schützen, und darauf vertrauen, dass der Systemadministrator ihre Daten geheim hält.
    Der Datenschutz konnte immer vom Administrator außer Kraft gesetzt werden, wenn er das Prinzip des Datenschutzes
    gegen andere Belange abwog, oder auf Geheiß seiner Vorgesetzten.
    Dann wurde starke Verschlüsselung für die breite Öffentlichkeit verfügbar, und Vertrauen war nicht mehr
    erforderlich.
    Daten konnten so gesichert werden, dass es für andere physisch unmöglich war, auf sie zuzugreifen,
    egal aus welchem Grund, egal wie gut die Rechtfertigung war, egal was.

    Es ist an der Zeit, dass wir das Gleiche auch für Geld nutzen.
    Mit einer elektronischen Währung, die auf kryptografischen Beweisen basiert, ohne die Notwendigkeit, einem dritten
    Mittelsmann zu vertrauen, kann Geld sicher und Transaktionen unkompliziert sein.

    Einer der grundlegenden Bausteine für ein solches System sind digitale Signaturen.
    Eine digitale Münze enthält den öffentlichen Schlüssel ihres Besitzers.
    Um sie zu übertragen, signiert der Besitzer die Münze zusammen mit dem öffentlichen Schlüssel des nächsten
    Besitzers.
    Jeder kann die Signaturen überprüfen, um die Kette der Eigentümerschaft zu verifizieren.
    Das funktioniert gut, um den Besitz zu sichern, lässt aber ein großes Problem ungelöst: Doppelausgaben.
    Jeder Besitzer könnte versuchen, eine bereits ausgegebene Münze erneut auszugeben, indem er sie einem anderen
    Besitzer erneut signiert.
    Die übliche Lösung ist, dass ein vertrauenswürdiges Unternehmen mit einer zentralen Datenbank auf Doppelausgaben
    prüft, aber das führt nur zurück zum Treuhandmodell.
    In ihrer zentralen Position kann die Firma die Benutzer übergehen, und die Gebühren, die nötig sind, um die Firma zu
    finanzieren, machen Micropayments unpraktisch.

    Bitcoins Lösung ist die Verwendung eines Peer-to-Peer-Netzwerks zur Überprüfung auf Doppelausgaben.
    Kurz gesagt, funktioniert das Netzwerk wie ein verteilter Zeitstempel-Server, der die erste Transaktion stempelt,
    die eine Münze ausgibt.
    Es macht sich die Tatsache zunutze, dass Informationen leicht zu verbreiten, aber schwer zu unterdrücken sind.
    Details zur Funktionsweise finden Sie im Konzeptpapier unter http://www.bitcoin.org/bitcoin.pdf.

    Das Ergebnis ist ein dezentrales System ohne einen einzigen Angriffspunkt für Fehler.
    Die Benutzer besitzen die Krypto-Schlüssel für ihr eigenes Geld und führen direkt miteinander Transaktionen durch,
    mit Hilfe des P2P-Netzwerks, um Doppelausgaben zu verhindern.

    Satoshi Nakamoto
\end{document}