\documentclass[sigconf]{acmart}
\usepackage{graphicx}
\usepackage{hyperref}
\usepackage[main=ngerman, english]{babel}
\usepackage{xcolor}
\usepackage{wasysym}
\usepackage{siunitx}
\usepackage[utf8]{inputenc}
\usepackage[T1]{fontenc}
\usepackage[nolist]{acronym}


\graphicspath{ {./images/} }

\AtBeginDocument{
    \providecommand\BibTeX{{
        \normalfont B\kern-0.5em{\scshape i\kern-0.25em b}\kern-0.8em\TeX}}}

\acmYear{2021}
\acmConference[Aktuelle Themen der IT-Sicherheit WS21/22]{Hochschule für Angewandte Wissenschaften Hof, 06.12.2021}{06.12.2021}{Hof, Deutschland}

\def\hyph{-\penalty0\hskip0pt\relax}

\begin{document}
    \begin{acronym}[EEPROM]
    \acro{DoS}{Denial of Service}
    \acro{SHA-256}{Secure Hash Algorithm \num{2} mit \num{256} Bit}
\end{acronym}


    \title{Kryptowährungen}
    \subtitle{Einführung und praktische Ansätze}
    \author{Noah Lehmann}
    \email{noah.lehmann@hof-university.de}
    \affiliation{
        \institution{Hochschule für Angewandte Wissenschaften}
        \streetaddress{Alfons-Goppel-Platz 1}
        \city{Hof}
        \country{Deutschland}
        \postcode{95028}
    }

    \begin{abstract}
        In einer Zeit, in der das Finanzsystem wiederholt stark inflationäre Tendenzen entwickelt und das Konzept
        \emph{Web 3.0} immer mehr an Bedeutung gewinnt, bedarf es einer neuen Technologie, die durch die Ideen des
        \emph{Web 3.0} die Probleme des Finanzsystems zu lösen versucht.
        Hierfür entwickelte die nach wie vor anonyme Entität Satoshi Nakamoto 2008 erstmals ein funktionierendes Konzept
        für eine Kryptowährung namens \emph{Bitcoin}.
        \emph{Bitcoin} dezentralisiert sein Transaktionssystem und eliminiert somit alle Mittelmänner.
        Dadurch entsteht ein über die sogenannte \emph{Blockchain} abgesichertes Peer-to-Peer-Bargeld.
    \end{abstract}

    \keywords{Kryptowährungen, Bitcoin, Blockchain, Konsens-Protokoll, Dezentralität}

    \maketitle


    \section{Einführung}
    Aktuelle Entwicklungen der Finanzsysteme weltweit übersteigen die regulären Wachstumsraten weit und tendieren immer
mehr in Richtung hoher Inflationen.
Zudem kommen seit Jahren immer mehr Fälle gezielter Manipulationen von Währungen auf.
Genau diese Feststellungen waren schon zu Beginn des \num{21.} Jahrhunderts Grundlage der Diskussion über die Möglichkeit der Umsetzung
von dezentralen, möglichst digitalen Währungen, teils zu Zeiten, in denen das Internet noch großenteils auf
Informationsdarstellung limitiert war.
Diese Phase des Internets wird heute als Web 1.0 bezeichnet, das vom heute immer noch aktuellen Web 2.0 abgelöst wurde.
Das Konzept des Web 2.0 ist die Erweiterung des statischen Internets der Informationen im Web 1.0 um die Teilnahme seiner
Nutzer.
Jedoch wurde diese Flut neuer Informationen ebenfalls von Unternehmen genutzt, die diese Daten Sammeln, weiter nutzen
und vermarkten wollen.
Daraus entstanden ist ein stark zentralisiertes Internet, welches von wenigen großen Firmen betrieben und bestimmt wird -
ganz wie das Finanzsystem.
Aus dem Bedürfnis heraus, sich der Kontrolle zentraler Instanzen zu entziehen und die Verantwortung der betroffenen
Systeme wieder den Nutzern anzueignen ist die Idee des sogenannten Web 3.0 entstanden, das als völlig dezentrales Internet
beschrieben wird, in dem sämtlicher Einfluss den Nutzern statt einigen wenigen zentralen Instanzen zusteht.

In genau diesem Zug sind erste Ideen für digitale Währungen entstanden, welche später durch ihren Bezug zur Kryptografie
als Kryptowährungen bezeichnet wurden.
Um kurz einzuschränken, was genau die Idee einer Kryptowährung beinhaltet wird im Folgenden kurz eine Definition zitiert:

\begin{quote}
    Kryptowährungen sind digitale (Quasi-)Währungen mit einem meist dezentralen, stets verteilten und kryptografisch
    abgesicherten Zahlungssystem~\cite{kryptodefinition}.
\end{quote}

Auf die genaue Bedeutung der einzelnen Punkte soll in den folgenden Kapiteln genauer eingegangen werden, jedoch sollte
die in diesem Kapitel kurz angedeutete Geschichte der Finanzsysteme genauer dargestellt werden, um die Gründe für die
Entwicklung von Kryptowährungen besser verstehen zu können.

    \section{Bekannte Finanzsysteme}
    Um zu verstehen, warum das Konzept der Kryptowährungen relevant ist, müssen zuerst bekannte und veraltete
Währungen nach ihren Schwachstellen analysiert werden.
Im Folgenden werden einige bekannte Währungen und Bezahlsysteme kurz erläutert.

\subsection{Einfache Wertgegenstände}

Das grundlegende Konzept des Bezahlens ist das Austauschen von Gegenständen, die den jeweiligen Beteiligten den
gleichen Mehrwert bieten.
Um sich jedoch nicht auf das Tauschen von Gütern zu beschränken, wurden früh Wertgegenstände als Währung eingeführt,
die einen festgelegten Wert symbolisiert.
Diese Wertgegenstände mussten schwer reproduzierbar sein, um Fälschungen vermeiden zu können.
Zudem mussten sie leicht transportabel sein, um im täglichen Gebrauch praktikabel zu bleiben, was den Währungen einen Vorteil
gegenüber dem einfachen Tausch von Waren bat.
Um eine Bezahlung dennoch zweifelsfrei bestätigen zu können, mussten die genutzten Wertgegenstände leicht verifizierbar sein.

Beispiele für solche Wertgegenstände sind Muscheln und Perlen.
Im Falle der Perlen zeigt sich beispielhaft die Schwäche eines Wertgegenstandes im überregionalen oder sogar globalen Handel.
So nutzten europäische Händler im Handel auf dem afrikanischen Kontinent Glasperlen, welche für sie relativ einfach
zu reproduzieren waren, um sie dann gegen schwer zu reproduzierende Arbeitskraft in Form von afrikanischen Sklaven
zu tauschen.

Später nutzte man deshalb wert-stabilere Ressourcen wie Silber und Gold zum Tausch gegen Waren, wobei die Vereinigten Staaten von
Amerika sogar den gesamten Währungswert des Dollars mit Gold verknüpften.
Da dies nicht genug Flexibilität bat, löste man die Wertverknüpfung \num{1971} auf, was zum aktuellen Stand des Dollars führte.

\subsection{Fiat-Währungen}
\label{subsec:fiat}


\emph{Fiat} stammt vom lateinischen Wort \textit{fieri} und bedeutet \textit{Es werde} oder \textit{Es soll}.
In Verknüpfung mit Währungen bezeichnet der Zusatz \emph{Fiat} somit die innere Wertlosigkeit der Währung.
Der tatsächliche Wert wird von außen vorgegeben, was den Unterschied zu historischen Währungen darstellt, deren Wert
maßgeblich vom Angebot abhing.

Fiat-Währungen werden zentral gesteuert, was einige Vorteile mit sich bringt.
Die Einheiten der Währung, das Bargeld, kann leicht in Umlauf gebracht werden, getauscht werden und sehr einfach von
Fälschungen Unterschieden werden.
Zudem kann durch eine staatliche Steuerung der Währung ebenfalls die Wirtschaft gesteuert und manipuliert werden.
Satoshi Nakamoto beschreibt den Stand der aktuellen Fiat-Währungen jedoch folgendermaßen:

\begin{quotation}
    Der Zentralbank muss vertraut werden, dass sie die Währung nicht entwertet, aber die Geschichte der
    Fiat-Währungen ist voll von Verstößen gegen dieses Vertrauen\cite{bitcoin_announcement}.
\end{quotation}

Ein Beispiel für den Missbrauch der Währungen und somit der negativen Manipulation durch Zentralbanken und Staaten zeigt die Entwertung des
venezuelanischen Bolivar durch die korrupte Regierung in den Anfängen des einundzwanzigsten Jahrhunderts.
Auch die gezielte Manipulation des chinesischen Renminbi zur Senkung des Exportpreises der eigenen Waren ist ein
Beleg des Missbrauchs.

%todo Belege

Eine weitere Eigenschaft von Fiat-Währungen ist die Zentralisierung der potenziellen Fehlerquellen.
Fiat Währungen werden durch Banken verwaltet, welche sowohl große Mengen der Währung, als auch sensible Daten der
Kunden halten und somit zu attraktiven Angriffszielen werden.
Auch dieses Phänomen beschreibt Satoschi Nakamoto:
\begin{quotation}
    Wir müssen ihnen unsere Privatsphäre anvertrauen und darauf hoffen, dass sie unsere Konten nicht von Betrügern
    leerräumen lassen\cite{bitcoin_announcement}.
\end{quotation}
Auch hierfür gibt es ein Beispiel.
Der US-Amerikanische Finanzdienstleister \emph{Equifax} wurde \num{2017} Opfer eines Angriffs, in dem über \num{140} Millionen
Datensätze aktueller und vergangener Kunden offengelegt wurden.

%todo Belege

\subsection{Erste Ansätze für Kryptowährungen}

\emph{Wei Dei} veröffentlichte um die Jahrtausendwende ein Konzept namens \emph{b-Money}, in welchem er ein anonymes,
verteiltes und digitale Bargeld beschrieb.
\emph{B-Money} legte den Fokus des Geldes auf Dezentralität und ermöglichte das Ausführen von Transaktionen ohne
Mittelsmänner.

Wenige Jahre später entwickelte Adam Back \emph{Hashcash}, welches als \ac{DoS}-Schutz für Mail-Server entwickelt wurde.
Das Konzept beschreibt erstmals die Idee des \emph{Proof of Work} oder Arbeitsnachweises.
Mail-Clients mussten hier eine relativ aufwändige Rechenaufgabe lösen, um eine Mail an den Server schicken zu können.
Somit wurde eine Überflutung von Anfragen an den Server unterbunden.
Der Server konnte die Lösung der Rechenaufgabe sehr leicht verifizieren.
Dieses Konzept wurde später von Kryptowährungen wie \emph{Bitcoin} adaptiert.

    \section{Kryptowährungen}
    \label{sec:krypto}
    Im letzten Kapitel wurden einige Probleme von vergangenen Währungen und den aktuellen Fiat-Währungen erläutert,
darunter die Instabilität des Wertes, die Manipulierbarkeit und die Zentralität der Fehlerquellen.
Kryptowährungen versuchen diese Probleme zu eliminieren, erste Ansätze hierfür wurden ebenfalls im vorigen Kapitel
gezeigt.

Um zu verstehen, wie Kryptowährungen funktionieren, wird im folgenden Kapitel auf die Ziele von Kryptowährungen
und auf die Ansätze, wie man diese Ziele erreichen kann, eingegangen.
Folgende Ziele werden betrachtet:
\begin{itemize}
    \item Dezentralisierung
    \item Verteilter Konsens
    \item Kein Vertrauen
    \item Inflationssicherheit
    \item Synchronisation
    \item Stabilität
\end{itemize}

Um diese Ziele und deren Lösungen erklären zu können, wird im folgenden Kapitel eine Währung aufgebaut, die sich stark
an der Implementierung von Bitcoin~\cite{bitcoin_whitepaper} orientiert.

\subsection{Dezentralisierung}

Wie im Kapitel~\ref{subsec:fiat} bereits erläutert wurde, bieten zentralisierte Finanzsysteme einige
Manipulationsmöglichkeiten, sowohl durch Angriffe als auch durch Einflussnahme Dritter.
Beim Aufbau einer neuen Währung muss man somit als ersten Schritt die Dezentralität garantieren.

Um eine Währung betreiben zu können, benötigt man ein sogenanntes Hauptbuch - im Folgenden als \emph{Ledger}
bezeichnet.
Dieses hält alle Informationen zu aktuellen Kontoständen und vergangenen Transaktionen.
Somit lassen sich alle Änderungen nachvollziehen.
Abbildung~\ref{fig:zentral} zeigt eine mögliche Form eines Ledgers.

\begin{figure}
           \centering
           \includegraphics[width=0.4\textwidth]{/home/nlehmann/Code/anonymitaet_in_krypto/images/zentral}
           \caption{Zentrales Ledger}
           \label{fig:zentral}
\end{figure}

Hält man dieses Ledger nun nur zentral, ergeben sich alle Nachteile einer konventionellen Währung weshalb das Ledger
an alle Beteiligten verteilt werden muss.
Jede Änderung wird an alle Teilnehmer der Währung bekannt gegeben und von allen geprüft.
Somit lässt sich das Netzwerk wie in Abbildung~\ref{fig:dezentral} beschreiben.

\begin{figure}
    \centering
    \includegraphics[width=0.4\textwidth]{/home/nlehmann/Code/anonymitaet_in_krypto/images/dezentral}
    \caption{Verteiltes Ledger}
    \label{fig:dezentral}
\end{figure}

Durch die Verteilung des Ledgers wurden nun die Grundlagen einer dezentralen Währung bereitgestellt.

\subsection{Verteilter Konsens}
\label{subsec:verteilung}

Die Verteilung des Ledgers funktioniert bei einem kleinen Netzwerk sehr gut. 
Sobald man aber skaliert und ein beliebig großes Netzwerk aufbaut, wird die Kommunikation ineffizienter und es
bieten sich klare Angriffs- und Manipulationsmöglichkeiten. 
Einige Teilnehmer des Netzwerkes könnten so gefälschte Transaktionen in das Ledger schreiben und dieses an neue Teilnehmer
kommunizieren. 
Eine naive Lösung dessen ist das Prüfen und Schreiben durch einen vertrauenswürdigen Teilnehmer, wie in Abbildung
~\ref{fig:schreiber} gezeigt.

\begin{figure}
    \centering
    \includegraphics[width=0.4\textwidth]{/home/nlehmann/Code/anonymitaet_in_krypto/images/schreiber}
    \caption{Vertrauenswürdiger Schreiber}
    \label{fig:schreiber}
\end{figure}

Dieser vertrauenswürdige Schreiber prüft nun alle Änderungen, die an ihn kommuniziert werden, schreibt diese 
gegebenenfalls in das Ledger und kommuniziert dies an alle anderen Teilnehmer, die nur Änderungen des gewählten Schreibers annehmen.
Dies bedarf ein grundlegendes Vertrauen in den Schreiber, bietet aber wiederum neue Angriffsmöglichkeiten.
Der Schreiber könnte entweder manipuliert, bedroht oder ersetzt werden, ohne dass die restlichen Teilnehmer
im Netzwerk etwas bemerken. 
In Kapitel~\ref{subsec:keinvertrauen} wird gezeigt, wie man einen vertrauenswürdigen Schreiber in einem dezentralen
System festlegen kann, ohne ihn angreifbar zu machen.

\subsection{Kein Vertrauen}
\label{subsec:keinvertrauen}

Wie bereits beschrieben, bietet ein einziger vertrauenswürdiger Schreiber eine Schwachstelle in einem Währungssystem.
Da das große Ziel einer Kryptowährung die Gleichberechtigung aller Teilnehmer ist, muss man den Vertrauensaspekt beim
Schreiben in das Ledger entfernen. 
Der Gleichberechtigung geschuldet müssen alle Teilnehmer im Netzwerk der Währung schreiben können und dürfen.
Hierfür kann man eine Lotterie erstellen, welche das Schreibrecht vergibt und an der grundsätzlich alle Teilnehmer des
Netzwerks teilnehmen können.
Diese Lotterie muss mindestens folgende Bedingungen erfüllen:
\begin{itemize}
    \item \textbf{Zufälligkeit der Schreibberechtigung}:
    Gegenstand der Verlosung ist die Schreibberechtigung, welche vollkommen unvorhersehbar erteilt werden muss.
    Wie in Kapitel~\ref{subsec:verteilung} bereits beschrieben, bietet ein festgelegter Schreiber eine
    Schwachstelle.
    Legt man diesen Schreiber jedoch zufällig unter allen Teilnehmern fest, ist es Angreifern unmöglich, den richtigen
    Teilnehmer zu finden.
    \item \textbf{Kosten für Teilnahme}:
    Um es Teilnehmern unmöglich zu machen, ihre Gewinnchancen zu weit zu erhöhen, muss die Teilnahme an der Lotterie
    etwas kosten.
    Somit bleibt die Zufälligkeit des Schreibers und die Chancengleichheit garantiert.
    \item \textbf{Gewinner-Bestimmung ohne zentrale Lotterieleitung}:
    Die Lotterie stellt sich den gleichen Herausforderungen wie die Währung selbst, sie muss völlig dezentral und somit
    nicht manipulierbar bleiben.
    Es muss eine Möglichkeit gefunden werden, den Gewinner dezentral zu bestimmen, sodass jeder Teilnehmer dies
    verifizieren und somit den Schreiber akzeptieren kann.
\end{itemize}

\begin{figure}
    \centering
    \includegraphics[width=0.4\textwidth]{/home/nlehmann/Code/anonymitaet_in_krypto/images/lotterie_ohne}
    \caption{Einrichten einer Lotterie}
    \label{fig:lotterie_ohne}
\end{figure}

Für dieses Problem lässt sich das Hash-Verfahren nutzen, denn es bietet den Vorteil, dass das Ergebnis eines Hashes
unabhängig von der Eingabe ist, jedoch bei mehrmaliger Ausführung des Hashes immer dasselbe ist.
Abbildung~\ref{fig:lotterie_ohne} zeigt eine mögliche Nutzung des Hash-Verfahrens in der \ac{SHA-256} Implementierung,
welche in der Umsetzung von Bitcoin an dieser Stelle verwendet wird.
Die Lotterie läuft folgendermaßen ab:
\begin{enumerate}
    \item Vor dem Lotterie-Durchgang wird ein gültiger Teilbereich im Wertebereich des Hash-Verfahrens festgelegt.
    \item Alle Transaktionen, die geschrieben werden müssen, werden von allen Teilnehmern gesammelt und verifiziert.
    \item Sobald eine Runde der Lotterie beginnt, werden alle gültigen Transaktionen mit einer zufällig gewählten Zahl -
    \emph{Nonce} - gehashed.
    \item Liegt der resultierende Hash nicht im gültigen Wertebereich, so wird eine neue Nonce gewählt und Schritt \num{3} wiederholt.
    \item Liegt der resultierende Hash im gültigen Wertebereich, so teilt der Teilnehmer seine Nonce an das Netzwerk mit.
    \item Alle anderen Teilnehmer im Netzwerk prüfen das Ergebnis und akzeptieren den Gewinner gegebenenfalls als Schreiber.
\end{enumerate}
Da das Finden der korrekten Nonce durch die Zufälligkeit der Hash-Ergebnisse beliebig viel Rechenkraft kosten kann, sind die Kosten für die Teilnahme garantiert.
Die Unvorhersehbarkeit des Hash-Ergebnisses garantiert wiederum die Zufälligkeit der Schreibberechtigung und das
einfache Prüfen des gültigen Hash-Werts durch das einmalige Ausführen des Hashes mit der gültigen Nonce ermöglicht eine
dezentrale Prüfung des Ergebnisses und macht somit eine zentrale Lotterieleitung überflüssig.
Die Kosten des Hash-verfahrens bieten den weiteren Vorteil, dass eine Manipulation des Ledgers durch den Schreiber nicht rentabel ist.
Die restlichen Teilnehmer können die Änderungen sehr einfach prüfen und im Notfall ablehnen, womit der Schreiber
sämtliche investierte Rechenkraft verschwendet hätte.

\subsection{Inflationssicherheit}

Es wurde nun eine Lotterie eingerichtet, welche den Schreiber für das Ledger vollkommen unvorhersehbar bestimmt.
Jedoch bedarf es für den Gewinner noch eine Art Preis, um ihn für die aufgewandte Rechenkraft zu entlohnen und einen
Anreiz für die Teilnahme an der Lotterie zu schaffen.

Das kann durch die sogenannte \emph{Coinbase}-Transaktion geschafft werden.
Diese wird von allen Teilnehmern der Lotterie zusätzlich zu allen anderen Transaktionen einbezogen und beschreibt den
Transfer eines Betrages der Währung an den Teilnehmer selbst.
Diese Transaktion benötigt keinen Ursprung, da sie neue Einheiten der Währung produziert.
Somit wird der Gewinner der Lotterie nicht nur entlohnt, es wird auch ein Weg festgelegt, wie neue Währung in den
Umlauf gebracht wird.

\begin{figure}
    \centering
    \includegraphics[width=0.4\textwidth]{/home/nlehmann/Code/anonymitaet_in_krypto/images/halving}
    \caption{Verlauf der Coinbase-Transaktionen}
    \label{fig:halving}
\end{figure}

Es muss nun nur der Wert dieser Coinbase-Transaktion festgelegt werden.
Dieser darf nicht konstant bleiben, da die Währung sonst einer konstanten Inflation ausgesetzt wird und somit nicht stabil ist.
Abbildung~\ref{fig:halving} zeigt das sogenannte \emph{Halving}, ein Mechanismus, der den Wert der Coinbase-Transaktion
nach einer bestimmten Anzahl an Lotterie-Durchgängen halbiert, bis er den Wert \num{0} erreicht hat.
Danach wird der Gewinner der Lotterie durch Transaktionsgebühren entlohnt, was einen weiteren Anreiz zur Teilnahme bietet.
Somit ist die Menge der Währung finalisiert, was die Währung an sich von Inflationstendenzen befreit.

\subsection{Synchronisation}

Es wurde nun eine Währung geschaffen, deren Ledger dezentral von allen Teilnehmern gehalten wird.
Eine Sammlung von Transaktionen wird von einem durch die Lotterie zufällig gewählten Schreiber festgehalten und von allen
Nutzern verifiziert.
Die Manipulation des Ledgers wird durch die Kosten des Hash-Verfahrens unterbunden, die Kosten hingegen werden durch
Transaktionsgebühren und die Coinbase-Transaktion gerechtfertigt.

Was noch nicht geklärt ist, ist wie genau das Ledger gesichert wird, denn neue Mitglieder im Netzwerk könnten nach wie
vor mit korrumpierten Versionen manipuliert werden.
Hier kommt die sogenannte \emph{Blockchain} ins Spiel.
Wie der Name vermuten lässt ist die Blockchain lediglich die Aneinanderreihung von Blöcken, wobei ein Block durch alle
folgenden Blöcke abgesichert wird.
Ein Block enthält die zu sichernden Informationen, im Falle der Kryptowährung die Transaktionen, den Hash des vorherigen
Blockes und den Arbeitsnachweis des Schreibers - die sogenannte \emph{Proof of Work}.
Ausgenommen ist hier der erste Block der Blockchain, auch \emph{Genesis-Block} genannt, welcher keine Informationen
zum vorigen Block enthält\footnote{Siehe Abbildung~\ref{fig:blockchain}}.

\begin{figure}
    \centering
    \includegraphics[width=0.4\textwidth]{/home/nlehmann/Code/anonymitaet_in_krypto/images/blockchain}
    \caption{Schema einer Blockchain}
    \label{fig:blockchain}
\end{figure}

Der Hash des vorigen Blocks garantiert die Unveränderbarkeit des Blocks, denn um nun einen älteren Block zu manipulieren,
müsste ein Angreifer die Informationen des Blocks so manipulieren, dass derselbe Hash entsteht, also eine Kollision
verursachen, was bei heutigen Hash-Verfahren quasi unmöglich ist.
%todo Nachweis
Eine Manipulation eines Blocks würde somit alle nachfolgenden Blöcke unbrauchbar machen.
Somit muss die Hash-Eingabe in der Lotterie noch um den Hash des vorigen Blockes erweitert werden, dies ist in Abbildung
~\ref{fig:lotterie} dargestellt.

\begin{figure}
    \centering
    \includegraphics[width=0.4\textwidth]{/home/nlehmann/Code/anonymitaet_in_krypto/images/lotterie}
    \caption{Einbinden der Blockchain in Lotterie}
    \label{fig:lotterie}
\end{figure}

Der Gewinner der Lotterie macht nun nicht einfach Einträge in ein Ledger, er schreibt stattdessen einen neuen Block in
die Blockchain, welche fortan als Ledger angesehen werden kann.

\subsection{Stabilität}

Im Grunde ist die Währung nun komplett.
Es wurde ein dezentrales Ledger durch die Blockchain erschaffen, welches nicht mehr manipuliert werden kann und in das
neue Einträge - oder Blöcke - von zufällig gewählten Schreibern geschrieben werden.
In diesem Ablauf wird die Skalierung jedoch noch nicht einbezogen, denn in einem beliebig großen Netzwerk von Teilnehmern,
die ständig miteinander kommunizieren, muss die Latenz bei der Übertragung von Nachrichten mit beachtet werden.

Nimmt man also ein weltweites Netzwerk, so kann es in der Verlosung der Schreibrechte passieren, dass in unterschiedlichen
Teilen des Netzwerkes unterschiedliche Blöcke gefunden werden, die alle gültig sind.
Grund hierfür sind Transaktionen, die nähere Knoten im Netzwerk eher erreichen, als weiter entfernte.
Weit entfernte Teilnehmer im Netzwerk starten ihren Versuch in der Lotterie also mit unterschiedlichen Transaktionen.
Finden nun mehrere Teilnehmer im Netzwerk gleichzeitig gültige Blöcke, in denen unterschiedliche Transaktionen gehalten
werden, bevor diese Transaktionen und die Blöcke von allen Teilnehmern akzeptiert oder gar empfangen wurden, so kommt es
zu sogenannten \emph{Blockkollisionen}.
Hier akzeptieren alle Teilnehmer jeweils den gültigen Block, dessen Transaktionen sie bereits kannten und der sie als erstes
erreicht hat.
Es kann also zu Zeitpunkten kommen, in denen verschiedene Blöcke in der Blockchain akzeptiert werden.
Um nun ein Spalten der Blockchain zu vermeiden und einen allgemeinen Konsens herzustellen, muss ein weiteres Konsens-Verfahren eingeführt werden.

Im Falle von Bitcoin heißt dieser Konsens \emph{Nakamoto-Konsens}.
Dieser besagt, dass die Blockchain mit dem höchsten Arbeitsaufwand - \emph{Proof of Work} - akzeptiert wird.
Findet somit ein Teilnehmer einen weiteren Block und teilt ihn an alle anderen Teilnehmer mit, bevor jemand anderes einen
Block findet, so wird dessen darunterliegende Blockchain als neue Grundlage akzeptiert, da in diese am meisten Aufwand
geflossen ist.
Alle anderen Blöcke werden verworfen.
Abbildung~\ref{fig:kollision} zeigt diesen Ablauf beispielhaft.
Hier kollidieren die Blöcke \emph{B} und \emph{C}, wobei Block \emph{D} Block \emph{C} zuerst bestätigt.
Block \emph{B} wird nun verworfen.
Verworfene Transaktionen gehen allerdings nicht verloren, sie werden in einem späteren Block neu aufgenommen und bis dahin
mit neuen Transaktionen im sogenannten \emph{Transaktionspool} gehalten, den alle Nutzer bei sich halten.

\begin{figure}
    \centering
    \includegraphics[width=0.4\textwidth]{/home/nlehmann/Code/anonymitaet_in_krypto/images/kollision}
    \caption{Nagamoto Konsens}
    \label{fig:kollision}
\end{figure}

Es wurde nun ein Protokoll geschaffen, welches ein \emph{dezentrales Ledger} durch eine \emph{Blockchain} realisiert, die jeder Nutzer
der Währung halten und verifizieren kann.
Alle Teilnehmer des Netzwerks sind gleichberechtigt und dürfen somit neue Blöcke schreiben, Voraussetzung für eine
Schreibberechtigung ist jedoch das Finden einer gültigen Zufallszahl - der \emph{Nonce}, welche, sobald sie gefunden wurde,
als Arbeitsnachweis - \emph{Proof of Work} - von allen Teilnehmer akzeptiert wird.
Die Kosten der Berechnung machen Manipulationsversuche unwirtschaftlich, da diese leicht erkannt und abgelehnt werden
können.
Für den Aufwand des Hashens wird der Schreiber mit der \emph{Coinbase-Transaktion} und Transaktionsgebühren belohnt.
Die \emph{Coinbase-Transaktion} reguliert ebenfalls die Gesamtmenge der Währungseinheiten, da sie nach einer bestimmten
Anzahl an geschriebenen Blöcken im Wert halbiert wird.
Das verhindert eine Inflation der Währung.
Sollten Blöcke, durch die Größe des Netzwerks geschuldet, kollidieren, so gilt die Blockchain mit dem höchsten Arbeitsaufwand
als neue akzeptierte Blockchain.
Alle anderen Blöcke werden verworfen, die Transaktionen werden in späteren Blöcken festgehalten.



    \section{Wallets und Transaktionen}
    In Kapitel~\ref{sec:krypto} wurde gezeigt, wie eine Kryptowährung aufgebaut werden kann, um einige Probleme von klassischen
Währungen zu eliminieren.
Im Folgenden sollen noch einmal kurz die Vorteile der Nutzung einer Kryptowährung gezeigt werden, um danach genauer
auf die Nutzung einer solchen Währung einzugehen.

Grundsätzlich implementieren die meisten Kryptowährungen die grundlegenden Ideen des \emph{Web 3.0}, konkret folgende:

\begin{itemize}
    \item \textbf{Unabhängigkeit von zentralen Instanzen}:
    Kryptowährungen geben die Verantwortung der Währung den Nutzern, statt sie zentralen Instanzen anzuvertrauen und sich
    Manipulationen durch diese und Angriffen auf diese auszusetzen.
    \item \textbf{Kryptografische Absicherung}:
    Kryptowährungen basieren in keiner Form auf Vertrauen, sondern stets auf kryptografischer Absicherung und der quasi
    Unmöglichkeit, die Verfahren zu umgehen, um die Währung zu manipulieren.
    \item \textbf{Privatsphäre}:
    Kryptowährungen verzichten in der Regel auf Identifizierungen der Nutzer und verfolgen die Ansicht, dass die Nutzung
    einer Währung unter die zu schützende Privatsphäre einzuordnen ist.
    Durch den Verzicht der Speicherung von sensiblen Daten machen sich Kryptowährungen auch nicht im Sinne der Privatsphäre
    der Nutzer angreifbar.
    \item \textbf{Wertstabilität}:
    Durch eine kontrollierte Ausschüttung der Währungseinheiten und die Limitierung dieser entziehen sich Kryptowährungen
    von jeglicher Inflation, Deflation oder vergleichbaren Wertinstabilitäten.
    Sie sind somit langfristig planbar, zuverlässig und vor Wertmanipulation geschützt.
\end{itemize}

Durch diese Eigenschaften heben sich Kryptowährungen von aktuellen \emph{Fiat}-Währungen ab und bieten somit eine
Alternative.
Im Folgenden soll genauer darauf eingegangen werden, wie Kryptowährungen genutzt werden können und wie die Nutzung konkret
implementiert sein könnte.
Hierfür wird erneut die Kryptowährung \emph{Bitcoin} als Beispiel verwendet.

\subsection{Krypto-Wallets}
\label{subsec:wallets}

Wie bereits angesprochen versuchen Kryptowährungen die Identifizierbarkeit aus der Währung zu entfernen.
Primärer Fokus liegt hier bei der Eliminierung persönlicher Daten aus der eindeutigen Zuordnung des Nutzers.
Die Herausforderung liegt also darin, eine Art Konto für eine Währung anzulegen, für das man sich nicht bei einer zentralen
Bank identifizieren muss und das nicht von anderen Nutzern oder Angreifern manipuliert werden kann.

Auch hierfür bedienen sich gängige Kryptowährungen der Kryptografie, indem sie zur eindeutigen Identifizierung eines Nutzers
asymmetrische Verschlüsselungsverfahren anwenden.

\begin{figure}
    \centering
    \includegraphics[width=0.4\textwidth]{/home/nlehmann/Code/anonymitaet_in_krypto/images/keygeneration}
    \caption{Erstellung eines Key-Pairs}
    \label{fig:key}
\end{figure}

In Abbildung~\ref{fig:key} wird anschaulich dargestellt, wie Nutzer beliebig viele Schlüsselpaare erzeugen können, aus
denen dann Adressen zur Identifizierung erzeugt werden können.

\begin{enumerate}
    \item Erstellen einer \emph{Seed-Phrase} aus menschenlesbaren Wörtern
    \item Hashing der Seed-Phrase zum Erstellen des Seeds
    \item Ableitung beliebig vieler Schlüsselpaare zur Adressgenerierung und weiteren Nutzung
\end{enumerate}

Hierbei wird Schritt 1 in vielen gängigen Kryptowährungen angewandt, um das Merken des Seeds zu vereinfachen.
Ein Nutzer kann somit alle seine Schlüsselpaare regenerieren, indem er sich die Wörter der Seed-Phrase merkt.

Aus einem Schlüsselpaar kann nun eine Adresse erzeugt werden, wie Abbildung~\ref{fig:adresse} zeigt.
Diese Adresse können nun der Nutzer und alle anderen Teilnehmer nutzen, um aus der Blockchain alle Transaktionen mit
dieser Adresse auszulesen.
Durch diese Transaktionen kann dann der Kontostand des Nutzers ermittelt werden, wodurch spätere Transaktionen validiert
werden können.
Abbildung~\ref{fig:adresse} zeigt in folgenden Schritten, wie eine Bitcoin Adresse aus einem Schlüsselpaar generiert
werden kann:

\begin{figure}
    \centering
    \includegraphics[width=0.4\textwidth]{/home/nlehmann/Code/anonymitaet_in_krypto/images/adresse}
    \caption{Erstellung einer Adresse}
    \label{fig:adresse}
\end{figure}

\begin{enumerate}
    \item Doppel-Hash des Public-Keys zur Verkleinerung dessen auf einen Wertebereich von \num{160} Bit
    \item Kodierung des Hashes zu \emph{Base58}
\end{enumerate}

Der Doppel-Hash in Schritt \num{1} ist in diesem konkreten Fall eine redundante Vorsichtsmaßnahme, um die Wahrscheinlichkeit von
Kollisionen der Hashes weiter zu verringern.
Das eigentliche Ziel des Hashes ist es, den Public-Key deutlich zu verkleinern, um die Nutzung der späteren Adresse
zu vereinfachen.
Auch Schritt \num{2} dient lediglich der einfacheren Nutzung, da \emph{Base58} ähnlich scheinende Zeichen aus dem Zeichensatz entfernt.
Somit sollen Verwechslungen vermieden werden.
Im Grunde kann also der Public-Key zur Identifizierung eines Nutzers genutzt werden, in der Praxis wandelt man diesen
zur einfacheren Nutzung allerdings etwas ab.
Daraus resultieren die Wallet-Adressen.

Wie es bei asymmetrischen Verschlüsselungsverfahren gängig ist, kann nun der Public-Key des Schlüsselpaars an alle Nutzer im Netzwerk
weitergegeben werden.
Der Private-Key sollte dabei nie preisgegeben werden.

Führt man nun eine Transaktion aus, so signiert der Sender die Transaktion mit seinem Private-Key.
Somit ist garantiert, dass die Transaktion definitiv vom entsprechenden Nutzer angestoßen wurde.
Der Empfänger und alle anderen Teilnehmer im Netzwerk können diese Transaktion verifizieren, indem sie sie mit dem
Public-Key des Senders entschlüsseln.
Der Inhalt einer Transaktion besteht somit nur aus zwei Adressen - der des Senders und der des Empfängers - und dem Betrag,
der transferiert werden soll.
Somit ist die Datenmenge, die an alle Nutzer bekannt gegeben werden muss, ebenfalls auf ein Minimum beschränkt.
Abbildung~\ref{fig:transaktion} zeigt diesen Ablauf schematisch.

\begin{figure}
    \centering
    \includegraphics[width=0.4\textwidth]{/home/nlehmann/Code/anonymitaet_in_krypto/images/transaction}
    \caption{Transaktionsablauf}
    \label{fig:transaktion}
\end{figure}

\subsection{Arten von Wallets}

Wie in Kapitel~\ref{subsec:wallets} beschrieben ist eine Wallet nichts anderes als ein Werkzeug, welches für einen Nutzer
die Adressen, Schlüsselpaare und Transaktionsabläufe betreut.
Dabei limitieren sich Wallets keineswegs auf die Nutzung für nur eine Kryptowährung, stattdessen können sie beliebig komplex
ausfallen und verschiedene Kryptowährungen für einen Nutzer verwalten.
Tabelle~\ref{tab:wallets} zeigt verschiedene Implementierungsarten von Wallets für Kryptowährungen.
\begin{table}[]
    \begin{tabular}{@{}llll@{}}
        \toprule
        \multicolumn{2}{c}{\textbf{Hot Wallets}} & \multicolumn{2}{c}{\textbf{Cold Wallets}} \\ \midrule
        \multicolumn{2}{c}{Web-Anwendung} & \multicolumn{2}{c}{Hardware} \\
        \multicolumn{2}{c}{Mobile-Anwendung} & \multicolumn{2}{c}{Digital} \\
        \multicolumn{2}{c}{Desktop-Anwendung} & \multicolumn{2}{c}{Papier} \\ \midrule
        \multicolumn{1}{c}{\textcolor{green}{\textbf{+}}} &
        \begin{tabular}[c]{@{}l@{}}
            einfach\\ kostenfrei
        \end{tabular} &
        \multicolumn{1}{c}{\textcolor{green}{\textbf{+}}} &
        sehr sicher \\
        \multicolumn{1}{c}{\textcolor{red}{\textbf{--}}} & unsicher &
        \multicolumn{1}{c}{\textcolor{red}{\textbf{--}}} &
        \begin{tabular}[c]{@{}l@{}}
            teuer\\ schwerer Einstieg
        \end{tabular} \\
        \bottomrule
    \end{tabular}
    \caption{Arten von Krypto-Wallets}
    \label{tab:wallets}
\end{table}
Wallets kann man grundsätzlich in zwei Arten unterscheiden, \emph{Cold-Wallets} und \emph{Hot-Wallets}.
Man betrachtet Hot-Wallets im Allgemeinen als einfacher zu Nutzen und günstiger in der Anschaffung.
Da sie jedoch, wie der Name impliziert, immer am Netzwerk hängen, machen Hot-Wallets sich leicht angreifbar und sind
für die Nutzung mit großen Beträgen nicht zu empfehlen.

Cold-Wallets hingegen sind oft komplizierter in der Nutzung und oftmals auch teuer in der Anschaffung, sind aber deutlich
sicherer als Hot-Wallets, insofern sie korrekt genutzt werden.
Wie der Name schon impliziert, hängen Cold-Wallets nicht permanent am Netzwerk und sind somit in der Zeit, in der sie
physisch getrennt sind, nicht über das Netzwerk angreifbar.

\subsection{Arten von Kryptowährungen}

Da \emph{Bitcoin} allgemein als die erste funktionierende und tatsächlich implementierte Kryptowährung gilt unterteilt
man den Begriff Kryptowährung in der Regel in zwei Unterbereiche: \emph{Bitcoin} und \emph{Altcoin}.
Altcoin beschreibt alle Währungen außer \emph{Bitcoin} und wird selbst noch einmal in zwei Unterkategorien geteilt:\linebreak
Bitcoin\hyph{}Abwandlungen und Kryptowährungen mit nativen Blockchains.
Bitcoin\hyph{}Abwandlungen sind hierbei Währungen, welche lediglich Teile der Bitcoin Implementierung abgewandelt haben und deren
Blockchain in den frühen Blöcken identisch mit der von Bitcoin ist.
Währungen mit einer nativen oder eigenen Blockchain sind seit Beginn ihrer Blockchain unabhängig von Bitcoin entwickelt worden.
Abbildung~\ref{fig:coins} zeigt die Unterteilung von Kryptowährungen und gibt einige Beispiele für die Unterkategorien an.

\begin{figure}
    \centering
    \includegraphics[width=0.4\textwidth]{/home/nlehmann/Code/anonymitaet_in_krypto/images/waehrungen}
    \caption{Kryptowährungen}
    \label{fig:coins}
\end{figure}




    \section{Fazit}
    Die große Anzahl der konkreten Kryptowährungen deutet darauf hin, dass die Konzepte und Implementierungen dieser auch
funktionieren und von der Allgemeinheit akzeptiert werden. %todo wieviele Kryptos genau?
Stand jetzt bieten die größten drei Währungen einen umgewandelten Wert von ?Euro und sind somit eine echte Alternative
zu klassischen Währungen geworden.
Auch die Akzeptanz der Währung bei Dienstleistern wächst immer weiter.
Und neben den konzeptuellen Erfolgen von Kryptowährungen bieten sie ebenfalls Sicherheit für Menschen weltweit, die in der
Nutzung ihrer gängigen Fiat-Währung eingeschränkt sind, sei es durch Wertmanipulation oder Unterdrückung, da das Halten
von Kryptowährungen lediglich das Merken der Seed-Phrase aus Kapitel~\ref{subsec:wallets} erfordert und somit im
Extremfall kaum nachverfolgbar ist.
Jedoch sollte man sich immer bewusst sein, dass physischer Zugriff auf Wallets oder deren Benutzer immer ein großes Problem
sein kann.
Auch sollte man nie einem System vertrauen, dass man nicht versteht oder nicht geprüft hat, was die Einstiegshürde für
Kryptowährung deutlich erhöht.
Dennoch sind sie eine vielversprechende Alternative zu klassischen Systemen und fügen sich der Idee des Web 3.0 nahtlos
an.
Die Entwicklung von Kryptowährungen sollte in den kommenden Jahren mit Spannung weiterverfolgt werden.

    \nocite{*}
    \bibliographystyle{ACM-Reference-Format}
    \bibliography{paper}

    \appendix


\end{document}
