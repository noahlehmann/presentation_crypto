Aktuelle Entwicklungen der Finanzsysteme weltweit übersteigen die regulären Wachstumsraten weit und tendieren immer
mehr in Richtung hoher Inflationen.
Zudem kommen seit Jahren immer mehr Fälle gezielter Manipulationen von Währungen auf.
Genau diese Feststellungen waren schon zu Beginn des \num{21.} Jahrhunderts Grundlage der Diskussion über die Möglichkeiten
von dezentralen, möglichst digitalen Währungen, teils zu Zeiten, in denen das Internet noch großenteils auf
Informationsdarstellung limitiert war.
Diese Phase des Internets heute als Web 1.0 bezeichnet, das vom heute immer noch aktuellen Web 2.0 gefolgt wurde.
Das Konzept des Web 2.0 ist die Erweiterung des statischen Internet der Informationen im Web 1.0 um die Teilnahme seiner
Nutzer.
Jedoch wurde diese Flut neuer Informationen ebenfalls von Unternehmen genutzt, die diese Daten Sammeln und weiter nutzen
und vermarkten wollen.
Daraus entstanden ist ein stark zentralisiertes Internet, welches von wenigen großen Firmen betrieben und bestimmt wird -
ganz wie das Finanzsystem.
Aus dem Bedürfnis heraus, sich der Kontrolle zentraler Instanzen zu entziehen und die Verantwortung der betroffenen
Systeme wieder den Nutzern anzueignen ist die Idee des sogenannten Web 3.0 entstanden, das als völlig dezentrales Internet
beschrieben wird, in dem sämtlicher Einfluss den Nutzern statt einigen wenigen zentralen Instanzen zu steht.\\
In genau diesem Zug sind erste Ideen für digitale Währungen entstanden, welche später durch ihren Bezug zur Kryptografie
als Kryptowährungen bezeichnet wurden.
Um kurz einzuschränken, was genau die Idee einer Kryptowährung beinhaltet wird im Folgenden kurz eine Definition zitiert:

\begin{quote}
    Kryptowährungen sind digitale (Quasi-)Währungen mit einem meist dezentralen, stets verteilten und kryptografisch
    abgesicherten Zahlungssystem~\cite{kryptodefinition}.
\end{quote}

Über die genaue Bedeutung der einzelnen Punkte soll in den folgenden Kapitel genauer eingegangen werden, jedoch sollte
die in diesem Kapitel kurz angedeutete Geschichte der Finanzsysteme kurz dargestellt werden, um die Gründe für die
Entwicklung von Kryptowährungen besser verstehen zu können.