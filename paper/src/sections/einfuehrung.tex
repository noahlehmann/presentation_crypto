%% todo definition krypto
Um zu verstehen, warum das Konzept von Kryptowährungen relevant sein könnten, müssen zuerst bekannte und veraltete
Währungen nach ihren Schwachstellen analysiert werden.
Im Folgenden werden einige bekannte Währungen und Bezahlsysteme kurz erläutert.

\subsection{Einfache Wertgegenstände}

Das grundlegende Konzept des Bezahlens ist das Tauschen zweier Gegenstände, die den jeweiligen Beteiligten den
gleichen Mehrwert bieten.
Um sich jedoch nicht auf das Tauschen von Gütern zu beschränken wurden früh Wertgegenstände als Währung eingeführt,
die einen festgelegten Wert symbolisiert.
Diese Wertgegenstände mussten schwer reproduzierbar sein, um Fälschungen vermeiden zu können.
Zudem mussten sie leicht transportabel sein, um im täglichen Gebrauch praktikabel zu bleiben.
Das bat den Währungen einen Vorteil gegenüber dem einfachen Tausch von Waren.
Um eine Bezahlung bestätigen zu können, mussten diese Wertgegenstände leicht verifizierbar sein.\\

Beispiele hierfür sind Muscheln und Perlen.
Im Falle der Perlen zeigt sich die Schwäche eines Wertgegenstandes im überregionalen oder sogar globalen Handel.
So nutzten europäische Händler im Handel auf dem afrikanischen Kontinent Glasperlen, welche für sie relativ einfach
zu reproduzieren waren, um sie dann gegen schwer zu reproduzierende Arbeitskraft in Form von afrikanischen Sklaven
zu tauschen.\\

Später nutzte man stabilere Ressourcen wie Silber und Gold zum Tausch gegen Waren, wobei die Vereinigten Staaten von
Amerika sogar ihren gesamten Währungswert des Dollars mit Gold verknüpften.
Da dies nicht flexibel genug war, löste man die Wertverknüpfung 1971, was zum aktuellen Stand des Dollars führte.

\subsection{Fiat-Währungen}

%todo definition form
\textit{Fiat} stammt vom lateinischen Wort \textit{fieri} und bedeutet \textit{Es werde} oder \textit{Es soll}.
In Verknüpfung mit Währungen bezeichnet \textit{Fiat} somit die innere Wertlosigkeit der Währung.
Der tatsächliche Wert wird von außen vorgegeben, was den Unterschied zu historischen Währungen darstellt, deren Wert
maßgeblich vom Angebot abhing.\\

Fiat-Währungen werden zentral gesteuert, was einige Vorteile mit sich bringt.
Die Einheiten der Währung, das Bargeld, kann leicht in Umlauf gebracht werden, getauscht werden und sehr einfach von
Fälschungen Unterschieden werden.
Zudem kann durch eine staatliche Steuerung der Währung ebenfalls die Wirtschaft gesteuert und manipuliert werden.
Satoshi Nakamoto beschreibt den Stand der aktuellen Fiat-Währungen folgendermaßen:

\begin{quotation}
    Der Zentralbank muss vertraut werden, dass sie die Währung nicht entwertet, aber die Geschichte der
    Fiat-Währungen ist voll von Verstößen gegen dieses Vertrauen\cite{bitcoin_announcement}.
\end{quotation}

Ein Beispiel für den Missbrauch der Währungen von Zentralbanken und Staaten zeigt die Entwertung des
venezuelanischen Bolivars durch die korrupte Regierung in den anfängen des einundzwanzigsten Jahrhunderts.
Auch die gezielte Manipulation des chinesischen Renminbi zur Senkung des Exportpreises der eigenen Waren ist ein
Beleg des Missbrauchs.\\

Eine weitere Eigenschaft von Fiat-Währungen ist die zentralisierung der potenziellen Fehlerquellen.
Fiat Währungen werden durch Banken verwaltet, welche sowohl große Mengen der Währung, als auch sensible Daten der
Kunden halten und somit zu attraktiven Angriffszielen werden.
Auch dieses Phänomen beschreibt Satoschi Nakamoto:
\begin{quotation}
    Wir müssen ihnen unsere Privatsphäre anvertrauen und darauf hoffen, dass sie unsere Konten nicht von Betrügern
    leerräumen lassen\cite{bitcoin_announcement}.
\end{quotation}
Auch hierfür gibt es ein Beispiel.
Der US-Amerikanische Finanzdienstleister \textit{Equifax} wurde 2017 Opfer eines Angriffs, in dem über 140 Millionen
Datensätze aktueller und vergangener Kunden offengelegt wurden.

\subsection{Erste Ansätze für Kryptowährungen}

Wei Dei veröffentlichte um die Jahrtausendwende ein Konzept namens \textit{b-Money}, in welchem er ein anonymes,
verteiltes und digitale Bargeld beschrieb.
\textit{B-Money} legte den Fokus des Geldes auf Dezentralität und ermöglichte das Ausführen von Transaktionen ohne
Mittelsmänner.\\

%todo DoS Definition
Wenige Jahre später entwickelte Adam Back \textit{Hashcash}, welches als DoS-Schutz für Mail-Server entwickelt wurde.
Das Konzept beschreibt erstmals die Idee des \textit{Proof of Work} oder Arbeitsnachweises.
Mail Clients mussten hier eine relativ aufwändige Rechenaufgabe lösen, um eine Mail an den Server schicken zu können.
Somit wurde eine Überflutung von Anfragen an den Server unterbunden.
Der Server konnte die Lösung der Rechenaufgabe sehr leicht verifizieren.
Dieses Konzept wurde später von Kryptowährungen wie \textit{Bitcoin} adaptiert.