Im vergangenen Kapitel wurden einige Probleme von vergangenen Währungen und den aktuellen Fiat-Währungen erläutert,
darunter die Instabilität des Wertes, Manipulierbarkeit und die Zentralität der Fehlerquellen.
Kryptowährungen versuchen diese Probleme zu eliminieren, erste Ansätze hierfür wurden ebenfalls im vorigen Kapitel
gezeigt.\\
Um zu verstehen, wie Kryptowährungen funktionieren, wird im folgenden Kapitel auf die Ziele von Kryptowährungen
eingegangen und auf die Ansätze, wie man diese Ziele erreichen kann.
Folgende Ziele werden betrachtet:
\begin{itemize}
    \item Dezentralisierung
    \item Verteilter Konsens
    \item Kein Vertrauen
    \item Inflationssicherheit
    \item Synchronisation
    \item Stabilität
\end{itemize}
Um diese Ziele und deren Lösungen erklären zu können, wird im folgenden Kapitel eine Währung aufgebaut, die sich stark
an der Implementierung von Bitcoin~\cite{bitcoin_whitepaper} orientiert.

\subsection{Dezentralisierung}

Wie im Kapitel~\nameref{subsec:fiat} bereits erläutert wurde, bieten zentralisierte Finanzsysteme einige
Manipulationsmöglichkeiten, sowohl durch Angriffe als auch durch Einflussnahme Dritter.
Beim Aufbau einer neuen Währung muss man somit als erstes die Dezentralität garantieren.\\
Um eine Währung betreiben zu können, benötigt man ein sogenanntes Hauptbuch - im Folgenden als \textit{Ledger}
bezeichnet.
Dieses hält alle Informationen zu aktuellen Kontoständen und vergangenen Transaktionen.
Somit lassen sich alle Änderungen nachvollziehen.
Abbildung~\ref{fig:zentral} zeigt eine mögliche Form eines Ledgers.

\begin{figure}
           \centering
           \includegraphics[width=0.3\textwidth]{zentral}
           \caption{Zentrales Ledger}
           \label{fig:zentral}
\end{figure}

Hält man dieses Ledger nun nur zentral, ergeben sich alle Nachteile einer konventionellen Währung.
Somit muss das Ledger an alle Beteiligten verteilt werden.
Jede Änderung wird an alle Teilnehmer der Währung bekannt gegeben und von allen geprüft.
Somit lässt sich das Netzwerk wie in Abbildung~\nameref{fig:dezentral} beschreiben.

\begin{figure}
    \centering
    \includegraphics[width=0.3\textwidth]{dezentral}
    \caption{Verteiltes Ledger}
    \label{fig:dezentral}
\end{figure}

Durch die Verteilung des Ledgers wurden nun die Grundlagen einer dezentralen Währung bereitgestellt.

\subsection{Verteilter Konsens}
\label{subsec:verteilung}

Die Verteilung des Ledgers funktioniert bei einem kleinen Netzwerk sehr gut. 
Sobald man aber skaliert und ein beliebig großes Netzwerk aufbaut, wird die Kommunikation immer ineffizienter und es
bieten sich klare Angriffs- und Manipulationsmöglichkeiten. 
Einige Teilnehmer des Netzwerkes könnten so gefälschte Ausgaben in das Ledger schreiben und dieses an neue Teilnehmer 
kommunizieren. 
Eine naive Lösung dessen ist das Prüfen und Schreiben durch einen vertrauenswürdigen Teilnehmer, wie in Abbildung
~\nameref{fig:schreiber} gezeigt.

\begin{figure}
    \centering
    \includegraphics[width=0.3\textwidth]{schreiber}
    \caption{Vertrauenswürdiger Schreiber}
    \label{fig:schreiber}
\end{figure}

Dieser vertrauenswürdige Schreiber prüft nun alle Änderungen, die an ihn kommuniziert werden, schreibt diese 
gegebenenfalls in das Ledger und kommuniziert dies an alle anderen Teilnehmer, die nur Änderungen des Schreibers annehmen.
Dies bedarf ein grundlegendes Vertrauen in den Schreiber, bietet aber wiederum neue Angriffsmöglichkeiten.
Der Schreiber könnte entweder manipuliert, bedroht oder durch Dritte ersetzt werden, ohne dass die restlichen Teilnehmer
im Netzwerk etwas bemerken. 
In Kapitel~\nameref{subsec:keinvertrauen} wird gezeigt, wie man einen einen vertrauenswürdigen Schreiber in einem dezentralen
System festlegen kann, ohne ihn angreifbar zu machen.

\subsection{Kein Vertrauen}
\label{subsec:keinvertrauen}

Wie bereits beschrieben bietet ein einziger vertrauenswürdiger Schreiber eine Schwachstelle in einer neuen Währung. 
Da das große Ziel einer Kryptowährung die Gleichberechtigung aller Teilnehmer ist, muss man den Vertrauensaspekt beim
Schreiben in das Ledger entfernen. 
Der Gleichberechtigung geschuldet müssen alle Teilnehmer im Netzwerk der Währung schreiben können.
Hierfür kann man eine Lotterie erstellen, welche das Schreibrecht vergibt ud an der grundsätzlich alle Teilnehmer des 
Netzwerks teilnehmen können.
Diese Lotterie muss mindestens folgende Bedingungen erfüllen:
\begin{itemize}
    \item \textbf{Zufälligkeit der Schreibberechtigung}
    Gegenstand der Verlosung ist die Schreibberechtigung, welche vollkommen unvorhersehbar erteilt werden muss.
    Wie in Kapitel~\nameref{subsec:verteilung} bereits beschrieben bietet ein festgelegter Schreiber eine
    Schwachstelle.
    Legt man diesen Schreiber jedoch zufällig unter allen Teilnehmern fest, ist es Angreifern unmöglich, den richtigen
    Teilnehmer zu finden.
    \item \textbf{Kosten für Teilnahme}
    Um es Teilnehmern unmöglich zu machen, ihre Gewinnchancen zu weit zu erhöhen, muss die Teilnahme an der Lotterie
    etwas Kosten.
    Somit bleibt die Zufälligkeit des Schreibers und die Chancengleichheit garantiert.
    \item \textbf{Gewinner-Bestimmung ohne zentrale Lotterieleitung}
    Die Lotterie stellt sich den gleichen Herausforderungen wie die Währung selbst, sie muss völlig dezentral und somit
    nicht manipulierbar bleiben.
    Es muss eine Möglichkeit gefunden werden, den Gewinner dezentral zu bestimmen, so dass jeder Teilnehmer dies
    verifizieren und somit den Schreiber akzeptieren kann.
\end{itemize}

\begin{figure}
    \centering
    \includegraphics[width=0.4\textwidth]{lotterie_ohne}
    \caption{Einrichten einer Lotterie}
    \label{fig:lotterie_ohne}
\end{figure}

Für dieses Problem lässt sich das Hash-Verfahren nutzen, denn es bietet den Vorteil, dass das Ergebnis eines Hashes
unabhängig von der Eingabe ist, jedoch bei mehrmaliger Ausführung immer das selbe ist.
Abbildung~\nameref{fig:lotterie_ohne} zeigt eine mögliche Nutzung des Hash-Verfahrens in der SHA-256 Implementierung,
welche in der Umsetzung von Bitcoin an dieser Stelle verwendet wird.
Die Lotterie läuft folgendermaßen ab:
\begin{enumerate}
    \item Vor dem Lotterie-Durchgang wird ein gültiger Teilbereich im Wertebereich des Hash-Verfahrens festgelegt
    \item Alle Transaktionen, die geschrieben werden müssen, werden von allen Teilnehmern gesammelt und verifiziert.
    \item Sobald eine Runde der Lotterie beginnt, werden alle gültigen Transaktionen mit einer zufällig gewählten Zahl -
    \textit{Nonce} - gehashed.
    \item Liegt der resultierende Hash nicht im gültigen Wertebereich wird eine neue Nonce gewählt und Schritt 3 wiederholt
    \item Liegt der resultierende Hash im gültigen Wertebereich, so teilt der Teilnehmer seine Nonce an das Netzwerk mit.
    \item Alle anderen Teilnehmer im Netzwerk prüfen das Ergebnis und akzeptieren den Gewinner gegebenenfalls als Schreiber
\end{enumerate}

Da das finden der korrekten Nonce beliebig viel Rechenkraft kosten kann, sind die Kosten für die Teilnahme garantiert.
Die Unvorhersehbarkeit des Hash-Ergebnisses garantiert wiederum die Zufälligkeit der Schreibberechtigung und das
einfache Prüfen des gültigen Hashes durch das einmalige Ausführen des Hashes mit der gültigen Nonce ermöglicht eine
dezentrale Prüfung des Ergebnisses und macht somit eine zentrale Lotterieleitung überflüssig.

\subsection{Inflationssicherheit}

Es wurde nun eine Lotterie eingerichtet, welche den Schreiber für das Ledger vollkommen unvorhersehbar bestimmt.
Jedoch bedarf es für den Gewinner noch eine Art Belohnung, um ihn für die aufgewandte Rechenkraft zu entlohnen und einen
Anreiz für die Teilnahme an der Lotterie zu schaffen.\\
Das kann durch die sogenannte \textit{Coinbase}-Transaktion geschafft werden.
Diese wird von alles Teilnehmern der Lotterie zusätzlich zu allen anderen Transaktionen einbezogen und beschreibt den
Transfer eines Betrages der Währung an den Teilnehmer selbst.
Diese Transaktion benötigt keinen Ursprung, da sie neue Einheiten der Währung produziert.
Somit wird der Gewinner der Lotterie nicht nur entlohnt, es wird auch ein Weg festgelegt, wie neue Währung in den
Umlauf gebracht wird.\\

\begin{figure}
    \centering
    \includegraphics[width=0.3\textwidth]{halving}
    \caption{Verlauf der Coinbase-Transaktionen}
    \label{fig:halving}
\end{figure}

Es muss nun nur der Wert dieser Coinbase-Transaktion festgelegt werden.
Dieser darf auch nicht konstant bleiben, da die Währung sonst eine konstante Inflation erlebt und somit nicht stabil ist.
Abbildung~\nameref{fig:halving} zeigt das sogenannte Halving, ein Mechanismus, der den Wert der Coinbase-Transaktion
nach einer bestimmten Anzahl an Lotteriedurchgängen halbiert, bis er den Wert 0 erreicht hat.
Danach wird der Gewinner der Lotterie durch Transaktionsgebühren entlohnt, was einen weiteren Anreiz zur Teilnahme bietet.
Somit ist die Menge der Währung finalisiert, was die Währung an sich von Inflationstendenzen befreit.
