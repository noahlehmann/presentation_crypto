Im vergangenen Kapitel wurden einige Probleme von vergangenen Währungen und den aktuellen Fiat-Währungen erläutert,
darunter die Instabilität des Wertes, Manipulierbarkeit und die Zentralität der Fehlerquellen.
Kryptowährungen versuchen diese Probleme zu eliminieren, erste Ansätze hierfür wurden ebenfalls im vorigen Kapitel
gezeigt.\\
Um zu verstehen, wie Kryptowährungen funktionieren, wird im folgenden Kapitel auf die Ziele von Kryptowährungen
eingegangen und auf die Ansätze, wie man diese Ziele erreichen kann.
Folgende Ziele werden betrachtet:
\begin{itemize}
    \item Dezentralisierung
    \item Verteilter Konsens
    \item Kein Vertrauen
    \item Inflationssicherheit
    \item Synchronisation
    \item Stabilität
\end{itemize}
Um diese Ziele und deren Lösungen erklären zu können, wird im folgenden Kapitel eine Währung aufgebaut, die sich stark
an der Implementierung von Bitcoin~\cite{bitcoin_whitepaper} orientiert.

\subsection{Dezentralisierung}

Wie im Kapitel~\nameref{subsec:fiat} bereits erläutert wurde, bieten zentralisierte Finanzsysteme einige
Manipulationsmöglichkeiten, sowohl durch Angriffe als auch durch Einflussnahme Dritter.
Beim Aufbau einer neuen Währung muss man somit als erstes die Dezentralität garantieren.\\

Um eine Währung betreiben zu können, benötigt man ein sogenanntes Hauptbuch - im Folgenden als \textit{Ledger}
bezeichnet.
Dieses hält alle Informationen zu aktuellen Kontoständen und vergangenen Transaktionen.
Somit lassen sich alle Änderungen nachvollziehen.
Abbildung~\ref{fig:zentral} zeigt eine mögliche Form eines Ledgers.

\begin{figure}
           \centering
           \includegraphics[width=0.3\textwidth]{zentral}
           \caption{Zentrales Ledger}
           \label{fig:zentral}
\end{figure}