\usepackage{nameref}
\usepackage{biblatex}
\usepackage{textcomp}
\usepackage{csquotes} %Imports biblatex package
\usepackage{ifthen}
\usepackage{babel}
\usepackage{graphicx}
\usepackage{ulem}
\usepackage{xcolor}
\usepackage{booktabs}
\usepackage{multirow}

\addbibresource{main.bib} %Import the bibliography file
\graphicspath{ {./images/} }
%\usetheme{Dresden}
\useoutertheme{miniframes}
\beamertemplatenavigationsymbolsempty
%\setbeameroption{show notes}

\makeatletter
\newcommand*{\currentname}{\@currentlabelname}
\newcommand{\translation}[4]{\textbf{#1} \textrightarrow{} \textit{#2} (#3) \textendash{} \glqq #4\grqq}
\makeatother

\newboolean{sectiontoc}
\setboolean{sectiontoc}{true} % default true

% Gliederungstiefe
\setcounter{tocdepth}{1}

%Information to be included in the title page:
\title[Kryptowährungen]{Kryptowährungen}
\subtitle{Einführung und praktische Ansätze}
\author{Noah Lehmann}
\institute{Hochschule für Angewandte Wissenschaften Hof}
\date[06.12.2021]{Seminar Aktuelle Themen der IT-Sicherheit, 06.12.2021}

\begin{document}

    \AtBeginSection[]
    {
        \ifthenelse{\boolean{sectiontoc}}{
            \begin{frame}
                \frametitle{Zwischenstand}
                \tableofcontents[currentsection]
            \end{frame}
        }
    }

%==============================================================================
% Einführung
%==============================================================================

    \setboolean{sectiontoc}{false}
    \section{Einführung}
    Aktuelle Entwicklungen der Finanzsysteme weltweit übersteigen die regulären Wachstumsraten weit und tendieren immer
mehr in Richtung hoher Inflationen.
Zudem kommen seit Jahren immer mehr Fälle gezielter Manipulationen von Währungen auf.
Genau diese Feststellungen waren schon zu Beginn des \num{21.} Jahrhunderts Grundlage der Diskussion über die Möglichkeit der Umsetzung
von dezentralen, möglichst digitalen Währungen, teils zu Zeiten, in denen das Internet noch großenteils auf
Informationsdarstellung limitiert war.
Diese Phase des Internets wird heute als Web 1.0 bezeichnet, das vom heute immer noch aktuellen Web 2.0 abgelöst wurde.
Das Konzept des Web 2.0 ist die Erweiterung des statischen Internets der Informationen im Web 1.0 um die Teilnahme seiner
Nutzer.
Jedoch wurde diese Flut neuer Informationen ebenfalls von Unternehmen genutzt, die diese Daten Sammeln, weiter nutzen
und vermarkten wollen.
Daraus entstanden ist ein stark zentralisiertes Internet, welches von wenigen großen Firmen betrieben und bestimmt wird -
ganz wie das Finanzsystem.
Aus dem Bedürfnis heraus, sich der Kontrolle zentraler Instanzen zu entziehen und die Verantwortung der betroffenen
Systeme wieder den Nutzern anzueignen ist die Idee des sogenannten Web 3.0 entstanden, das als völlig dezentrales Internet
beschrieben wird, in dem sämtlicher Einfluss den Nutzern statt einigen wenigen zentralen Instanzen zusteht.

In genau diesem Zug sind erste Ideen für digitale Währungen entstanden, welche später durch ihren Bezug zur Kryptografie
als Kryptowährungen bezeichnet wurden.
Um kurz einzuschränken, was genau die Idee einer Kryptowährung beinhaltet wird im Folgenden kurz eine Definition zitiert:

\begin{quote}
    Kryptowährungen sind digitale (Quasi-)Währungen mit einem meist dezentralen, stets verteilten und kryptografisch
    abgesicherten Zahlungssystem~\cite{kryptodefinition}.
\end{quote}

Auf die genaue Bedeutung der einzelnen Punkte soll in den folgenden Kapiteln genauer eingegangen werden, jedoch sollte
die in diesem Kapitel kurz angedeutete Geschichte der Finanzsysteme genauer dargestellt werden, um die Gründe für die
Entwicklung von Kryptowährungen besser verstehen zu können.
    \setboolean{sectiontoc}{true}

%==============================================================================
% Geschichte
%==============================================================================


    \section{Geschichte}
    \begin{frame}{\currentname}{Teil I - Einfache Wertgegenstände}
    Womit hat man in der Vergangenheit bezahlt?\pause
    \begin{itemize}
        \item Wertgegenstände
        \begin{itemize}
            \item Schwer reproduzierbar
            \item Leicht transportabel
            \item Leicht verifizierbar
            \item Beispiele:
            \begin{itemize}
                \item Muscheln
                \item Perlen
                \item Silber
                \item Gold
            \end{itemize}
        \end{itemize}
    \end{itemize}
\end{frame}

\begin{frame}{\currentname}{Teil II - Währungen}
    \translation{Fiat}{fieri}{Latein}{es werde/ es soll}
    \begin{itemize}
        \item Zentrale Steuerung \pause
        \item Beeinflussung durch Staaten
    \end{itemize}
\end{frame}

\begin{frame}{Teil II - Währungen - Beeinflussung}
     \begin{quote}
         Der Zentralbank muss vertraut werden, dass sie die Währung nicht entwertet, aber die Geschichte der
         Fiat-Währungen ist voll von Verstößen gegen dieses Vertrauen~\cite{bitcoin_announcement}.
     \end{quote}
\end{frame}

\begin{frame}{\currentname}{Teil II - Währungen}
    \translation{Fiat}{fieri}{Latein}{es werde/ es soll}
    \begin{itemize}
        \item Zentrale Steuerung
        \item Beeinflussung durch Staaten
        \item Zentrale Fehlerquelle
    \end{itemize}
\end{frame}

\begin{frame}{Teil II - Währungen - Zentrale Fehlerquelle}
    \begin{quote}
        Wir müssen ihnen unsere Privatsphäre anvertrauen und darauf hoffen, dass sie unsere Konten nicht von
        Betrügern leerräumen lassen~\cite{bitcoin_announcement}.
    \end{quote}
\end{frame}

\begin{frame}{\currentname}{Teil III - Inspirationen für Bitcoin}
    \textbf{b-Money} (1999)
    \begin{itemize}
        \item digitale Währung
        \item Dezentralität und Transaktionen ohne Mittelsmänner
    \end{itemize}
    \textbf{Hashcash} (2002)
    \begin{itemize}
        \item DoS Schutz für Mail-Server
        \item \translation{PoW}{Proof of Work}{Englisch}{Arbeitsnachweis}
    \end{itemize}
\end{frame}

%==============================================================================
% Kryptowährung
%==============================================================================


    \section{Kryptowährung}
    \begin{frame}{\currentname}{P2P und Open-Source}
    \begin{quote}
        Ich habe ein neues Open-Source-P2P-E-Cash-System namens Bitcoin entwickelt.
    \end{quote}
    \pause
    \begin{itemize}
        \item Open-Source \textrightarrow Einsehbarer Quellcode \pause
        \item P2P \textrightarrow Peer-to-Peer, Direkte Kommunikation ohne Mittelsmänner \pause % todo Mittel(s)männer?
        \item E-Cash \textrightarrow Elektronisches Bargeld
    \end{itemize}
\end{frame}

\begin{frame}{\currentname}{Dezentralität und Vertrauen}
    \begin{quote}
        Es ist vollständig dezentralisiert, ohne zentralen Server oder Parteien, denen vertraut werden muss [\ldots].
    \end{quote}
    \begin{itemize}
        \item Dezentral \pause
        \begin{itemize}
            \item Kein zentraler Angriff möglich \pause
            \item Gleichberechtigung der Knoten
        \end{itemize} \pause
        \item Kein Vertrauen \textrightarrow Prüfe alles selbstständig
    \end{itemize}
\end{frame}

\begin{frame}{\currentname}{Kryptografische Mathematik}
    \begin{quote}
    [\ldots]
        da alles auf kryptografischen Beweisen statt auf Vertrauen basiert.
    \end{quote}
    \begin{itemize}
        \item Einwegfunktionen \pause
        \item Eliminiert Vertrauen
    \end{itemize}
\end{frame}

\begin{frame}{\currentname}{Privatsphäre und Sicherheit}
    \begin{quote}
        Wir müssen ihnen unsere Privatsphäre anvertrauen und darauf hoffen, dass sie unsere Konten nicht von Betrügern leerräumen lassen.
    \end{quote}
    \begin{itemize}
        \item Banken speichern Daten zentral \pause
        \item Zentrale Angriffsfläche
    \end{itemize}
\end{frame}

\begin{frame}{\currentname}{Fiat Währungen und Inflation}
    \begin{quote}
        Der Zentralbank muss vertraut werden, dass sie die Währung nicht entwertet, aber die Geschichte der Fiat-Währungen ist voll von Verstößen gegen dieses Vertrauen.
    \end{quote}
\end{frame}

\begin{frame}{\currentname}{Fiat Währungen und Inflation Teil II}
    \textbf{Fiat} \textrightarrow{} \textit{fiat} (Lateinisch) \textendash{} \glqq es werde\grqq \\
    Kein innerer Wert, von außen festgelegt. \pause
    \begin{itemize}
        \item Abwertung durch Inflation
        \item Beispiele
        \begin{itemize}
            \item Venezuela % todo Belege
            \item China
        \end{itemize}
    \end{itemize}
\end{frame}

\begin{frame}{\currentname}{Diebstahlschutz}
    \begin{quote}
        Daten konnten so gesichert werden, dass es für andere physisch unmöglich war, auf sie zuzugreifen, egal aus welchem Grund, egal wie gut die Rechtfertigung war, egal was.
        Es ist an der Zeit, dass wir das Gleiche auch für Geld nutzen.
    \end{quote}
\end{frame}

\begin{frame}{\currentname}{Diebstahlschutz Teil II}
    \begin{itemize}
        \item Konten lagern das Geld zentral \textrightarrow{} Vertrauen
        \item Staatliche Kontrolle (z.B.\ durch Europäische Zentralbank)
        \item Bankenstabilität von Wirtschaft abhängig % Griechenland
    \end{itemize}
\end{frame}

\begin{frame}{\currentname}{Diebstahlschutz Teil III}
    Wallets % todo
\end{frame}

\begin{frame}{\currentname}{Blockchain}
    \begin{quote}
        Bitcoins Lösung ist die Verwendung eines Peer-to-Peer-Netzwerks zur Überprüfung auf Doppelausgaben.
        Kurz gesagt, funktioniert das Netzwerk wie ein verteilter Zeitstempel-Server, der die erste Transaktion stempelt, die eine Münze ausgibt.
        Es macht sich die Tatsache zunutze, dass Informationen leicht zu verbreiten, aber schwer zu unterdrücken sind.
    \end{quote}
\end{frame}

\begin{frame}{\currentname}{Blockchain Teil II}
    % todo
\end{frame}

%------------------------------------------------------------------------------

%==============================================================================
% Anwendung
%==============================================================================


    \section{Anwendung}
    % Hier möglichst viel zeigen, Bsp BitCoin oder Ethereum

    \begin{frame}
        \frametitle{Wallet kaufen und einrichten}
    \end{frame}

    \begin{frame}
        \frametitle{BitCoin Node oder Lightnode zeigen}
    \end{frame}

    \begin{frame}
        \frametitle{BitCoin Kaufen}
    \end{frame}

    \begin{frame}
        \frametitle{Transaktionen}
    \end{frame}

    \begin{frame}
        \frametitle{Transaktionen nachverfolgen}
    \end{frame}

    \begin{frame}
        \frametitle{}
    \end{frame}

%==============================================================================
% Fazit
%==============================================================================


    \section{Fazit}
    Die große Anzahl der konkreten Kryptowährungen deutet darauf hin, dass die Konzepte und Implementierungen dieser auch
funktionieren und von der Allgemeinheit akzeptiert werden. %todo wieviele Kryptos genau?
Stand jetzt bieten die größten drei Währungen einen umgewandelten Wert von ?Euro und sind somit eine echte Alternative
zu klassischen Währungen geworden.
Auch die Akzeptanz der Währung bei Dienstleistern wächst immer weiter.
Und neben den konzeptuellen Erfolgen von Kryptowährungen bieten sie ebenfalls Sicherheit für Menschen weltweit, die in der
Nutzung ihrer gängigen Fiat-Währung eingeschränkt sind, sei es durch Wertmanipulation oder Unterdrückung, da das Halten
von Kryptowährungen lediglich das Merken der Seed-Phrase aus Kapitel~\ref{subsec:wallets} erfordert und somit im
Extremfall kaum nachverfolgbar ist.
Jedoch sollte man sich immer bewusst sein, dass physischer Zugriff auf Wallets oder deren Benutzer immer ein großes Problem
sein kann.
Auch sollte man nie einem System vertrauen, dass man nicht versteht oder nicht geprüft hat, was die Einstiegshürde für
Kryptowährung deutlich erhöht.
Dennoch sind sie eine vielversprechende Alternative zu klassischen Systemen und fügen sich der Idee des Web 3.0 nahtlos
an.
Die Entwicklung von Kryptowährungen sollte in den kommenden Jahren mit Spannung weiterverfolgt werden.

\end{document}