\begin{frame}{\currentname}{Motivation}
    \begin{itemize}
        \item Aktuelle Datenschutz- und Privatsphäre-Debatten %
        \item Anonymität gegen Datenmissbrauch % Werbung
        \item Suche nach nachvollziehbaren Währung
        \begin{itemize}
            \item Wertstabilität
            \item Nutzung
            \item Sicherung
        \end{itemize}
    \end{itemize}
    \textrightarrow{} Wie kann man ein Währung finden und nutzen, die obere Kriterien erfüllt?
    Zusätzliche Vorteile:
    \begin{itemize}
        \item Schutz vor Unterdrückung
        \item Schutz vor Manipulation
    \end{itemize}
\end{frame}

\begin{frame}{\currentname}{Datenschutz und Privatsphäre}
    Was kann potentiell bei Käufen gesammelt werden?
    \begin{itemize}
        \item Interessen
        \item Kaufverhalten
        \item Persönliche Daten (Einkommen, Alter, Wohnort, etc.)
    \end{itemize}
    Aktuell einzige Lösung:\pause~Bargeld\pause? % Problem -> Möglicherweise bargeldlose Zukunft?
\end{frame}

\begin{frame}{\currentname}{Datenschutz und Privatsphäre in Kryptowährungen}
    \begin{itemize}
        \item Transaktionen sind grundsätzlich P2P und pseudo-anonym % Nur die Adresse der Wallet is am Ende bekannt
        \item Kaufverhalten bei Transaktionen nicht bekannt, lediglich bei folgenden Schritten (Vertragsabschluss, Lieferung)
        \item TODO: \textdownarrow{} Community sehr anonym bedacht, Lieferung an Poststationen, etc
        \item Rückverfolgung der Blockchain und des Einkommens bei Verbindung zu Person sehr aufwändig und verwischbar
    \end{itemize}
\end{frame}

\begin{frame}{\currentname}{Währungen verstehen}
    \begin{itemize}
        \item Komplexität der Finanzen % grundsätzlich machbar
        \item Nachvollziehbarkeit von zentralen Entscheidungen nicht gegeben
        \item Unvorhersehbarkeit von Schwankungen (Börsen, Kursanpassung der Zentralbanken)
    \end{itemize}
\end{frame}

\begin{frame}{\currentname}{Kryptowährungen verstehen}
    \begin{itemize}
        \item Einstieg relativ einfach
        \item Mathematische/ kryptografische Grundkenntnisse reichen in der Regel
        \item Alle Vorgänge sind vorherbestimmt und Open-Source
        \item Geldmenge finalisiert
    \end{itemize}
\end{frame}

\begin{frame}{\currentname}{Kryptowährungen nutzen}
    \begin{itemize}
        \item Einstieg sehr einfach
        \item Kann beliebig komplex gestaltet werden
    \end{itemize}
\end{frame}

\begin{frame}{\currentname}{Kryptowährungen absichern}
    Angriffe auf die Währung allgemein:
    \begin{itemize}
        \item Double-Spend-Attacke
        \item Sybill-Attacken
        \item 51\%-Angriffe
        \item Verteilter Konsens
    \end{itemize}
    Angriffe auf das Kapital der Nutzer:
    \begin{itemize}
        \item Zugriff auf digitale Wallets
        \item Absicherung durch Entkopplung von Netzwerk
    \end{itemize}
\end{frame}

\begin{frame}{\currentname}{Exkurs} % Sicherung
    \begin{figure}
        \centering
        \includegraphics[width=0.8\textwidth]{five_dollar_wrench_attack}
        \caption{Fünf Dollar Schraubschlüssel Angriff~\cite{fivedollarattack}}
    \end{figure}
\end{frame}

\begin{frame}{\currentname}{Bonus: Schutz und Freiheit durch Kryptowährungen}
    \begin{itemize}
        \item
    \end{itemize}
\end{frame}