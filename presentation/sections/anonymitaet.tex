\begin{frame}{\currentname}{Beweggründe Teil I}
    Warum sollte man sich mit Kryptowährung beschäftigen?
    \begin{itemize}
        \item Investitionen
        \begin{itemize}
            \item Nur \glqq Leihe \grqq{} der Coins
            \item Lediglich Auszahlung der Wertsteigerung
            \item Wie Aktienhandel zu betrachten
        \end{itemize}
        \item Kauf von Kryptowährungen
        \begin{itemize}
            \item Inflationssicherheit
            \item Verteilung des eigenen Kapitals
        \end{itemize}
        \item Teilnahme im Netzwerk
        \begin{itemize}
            \item Selbständige Prüfung der Transaktionen
            \item Beitrag zur Sicherheit und Aufrechterhaltung
            \item %todo
        \end{itemize}
    \end{itemize}
\end{frame}

\begin{frame}{\currentname}{Beweggründe Teil II}
    Fragestellungen der folgenden Folien:
    \begin{itemize}
        \item Wie kann Anonymität in Kryptowährungen erreicht werden?
        \item Wie aufwändig ist die Aufrechterhaltung der Anonymität?
        \item Lohnt es sich, Anonymität im Umgang mit Kryptowährungen zu verfolgen?
    \end{itemize}
\end{frame}

\begin{frame}{\currentname}{Umgang mit Wallets und Transaktionen}
    - wie komme ich an Coins? (Anfangs und regelmäßig)
    - Wie kann ich sie halten (Arten der Wallets, Trennung zu Broker)
    - Wie sollte ich sie halten (5 Dollar Wrench Angriff?)
    %todo
\end{frame}

\begin{frame}{\currentname}{Exkurs}
    \begin{figure}
        \centering
        \includegraphics[width=0.8\textwidth]{five_dollar_wrench_attack}
        \caption{Fünf Dollar Schraubschlüssel Angriff~\cite{fivedollarattack}}
    \end{figure}
\end{frame}

\begin{frame}{\currentname}{Umgang mit Wallets und Transaktionen Teil II}
    - Was sollte ich bei Transaktionen allgemein beachten
    - Transaktionen an andere (ohne Verbindung zu eigener Identität)
    - Transaktionen an andere (Kauf von Gütern, Wechsel in Fiat oder Waren)
    - Verwischen von Spuren
    %todo
\end{frame}

\begin{frame}{\currentname}{Mögliche Schwachstellen}
    Eventuell schon in vorigen 2 Folien abgebildet
    %todo
\end{frame}

\begin{frame}{\currentname}{Aufwand der Nachverfolgung}
    Blockchain analyse
    Wie funktionieren Blockchain Reader
    Was braucht meine Wallet an Infos
    Was braucht ein möglicher "Angreifer" an Infos
    Wie aufwändig kann die Suche werden (Laufzeit)
    %todo
\end{frame}