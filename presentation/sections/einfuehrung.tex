\begin{frame}{Was sind Kryptowährungen?}
    \begin{quote}
        Kryptowährungen sind digitale (Quasi-)Währungen mit einem meist dezentralen, stets verteilten und kryptografisch abgesicherten Zahlungssystem\cite{kryptodefinition}.
    \end{quote}
    \pause
    Wie würde also eine Kryptowährung (wie BitCoin) aufgebaut werden?
\end{frame}
\note{
    Was fällt anderen bei "Kryptowährungen" ein? -> Schlagwörter sammeln
    ---
    DIGITAL -> nicht materiell, über das Internet nutzbar
    DEZENTRAL -> Nicht von Bank oder Staat kontrolliert
    VERTEILT -> Schlussfolgerung aus dezentral, Netzwerk von Nutzern
    KRYPTOGRAFISCH_ABGESICHERT -> Nicht manipulierbar
    ---
    Aufgrund der hohen Vielfalt Fokus auf BitCoin als Beispiel
}

\begin{frame}{\currentname}{Grundlegendes}
    \begin{itemize}
        \item Elektronisches Peer-to-Peer-Bargeld
        \item Keine zentrale Instanz, keine Mittelsmänner
        \item Inflationsbefreit
        \item (Dezentrales) Netzwerk aus Teilnehmern
    \end{itemize}
    Elektronisches Geld
    Bei Bezahlung ist Identität irrelevant (wie Bargeld)
    Nicht fälschbar
    etc. siehe erstes Kapitel
\end{frame}