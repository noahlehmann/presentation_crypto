\begin{frame}{\currentname}{P2P und Open-Source}
    \begin{quote}
        Ich habe ein neues Open-Source-P2P-E-Cash-System namens Bitcoin entwickelt.
    \end{quote}
    \pause
    \begin{itemize}
        \item Open-Source \textrightarrow Einsehbarer Quellcode \pause
        \item P2P \textrightarrow Peer-to-Peer, Direkte Kommunikation ohne Mittelsmänner \pause % todo Mittel(s)männer?
        \item E-Cash \textrightarrow Elektronisches Bargeld
    \end{itemize}
\end{frame}

\begin{frame}{\currentname}{Dezentralität und Vertrauen}
    \begin{quote}
        Es ist vollständig dezentralisiert, ohne zentralen Server oder Parteien, denen vertraut werden muss [\ldots].
    \end{quote}
    \begin{itemize}
        \item Dezentral \pause
        \begin{itemize}
            \item Kein zentraler Angriff möglich \pause
            \item Gleichberechtigung der Knoten
        \end{itemize} \pause
        \item Kein Vertrauen \textrightarrow Prüfe alles selbstständig
    \end{itemize}
\end{frame}

\begin{frame}{\currentname}{Kryptografische Mathematik}
    \begin{quote}
    [\ldots]
        da alles auf kryptografischen Beweisen statt auf Vertrauen basiert.
    \end{quote}
    \begin{itemize}
        \item Einwegfunktionen \pause
        \item Eliminiert Vertrauen
    \end{itemize}
\end{frame}

\begin{frame}{\currentname}{Privatsphäre und Sicherheit}
    \begin{quote}
        Wir müssen ihnen unsere Privatsphäre anvertrauen und darauf hoffen, dass sie unsere Konten nicht von Betrügern leerräumen lassen.
    \end{quote}
    \begin{itemize}
        \item Banken speichern Daten zentral \pause
        \item Zentrale Angriffsfläche
    \end{itemize}
\end{frame}

\begin{frame}{\currentname}{Fiat Währungen und Inflation}
    \begin{quote}
        Der Zentralbank muss vertraut werden, dass sie die Währung nicht entwertet, aber die Geschichte der Fiat-Währungen ist voll von Verstößen gegen dieses Vertrauen.
    \end{quote}
\end{frame}

\begin{frame}{\currentname}{Fiat Währungen und Inflation Teil II}
    \textbf{Fiat} \textrightarrow{} \textit{fiat} (Lateinisch) \textendash{} \glqq es werde\grqq \\
    Kein innerer Wert, von außen festgelegt. \pause
    \begin{itemize}
        \item Abwertung durch Inflation
        \item Beispiele
        \begin{itemize}
            \item Venezuela % todo Belege
            \item China
        \end{itemize}
    \end{itemize}
\end{frame}

\begin{frame}{\currentname}{Diebstahlschutz}
    \begin{quote}
        Daten konnten so gesichert werden, dass es für andere physisch unmöglich war, auf sie zuzugreifen, egal aus welchem Grund, egal wie gut die Rechtfertigung war, egal was.
        Es ist an der Zeit, dass wir das Gleiche auch für Geld nutzen.
    \end{quote}
\end{frame}

\begin{frame}{\currentname}{Diebstahlschutz Teil II}
    \begin{itemize}
        \item Konten lagern das Geld zentral \textrightarrow{} Vertrauen
        \item Staatliche Kontrolle (z.B.\ durch Europäische Zentralbank)
        \item Bankenstabilität von Wirtschaft abhängig % Griechenland
    \end{itemize}
\end{frame}

\begin{frame}{\currentname}{Diebstahlschutz Teil III}
    Wallets % todo
\end{frame}

\begin{frame}{\currentname}{Blockchain}
    \begin{quote}
        Bitcoins Lösung ist die Verwendung eines Peer-to-Peer-Netzwerks zur Überprüfung auf Doppelausgaben.
        Kurz gesagt, funktioniert das Netzwerk wie ein verteilter Zeitstempel-Server, der die erste Transaktion stempelt, die eine Münze ausgibt.
        Es macht sich die Tatsache zunutze, dass Informationen leicht zu verbreiten, aber schwer zu unterdrücken sind.
    \end{quote}
\end{frame}

\begin{frame}{\currentname}{Blockchain Teil II}
    % todo
\end{frame}

%------------------------------------------------------------------------------