Die große Anzahl der konkreten Kryptowährungen deutet darauf hin, dass die Konzepte und Implementierungen dieser auch
funktionieren und von der Allgemeinheit akzeptiert werden. %todo wieviele Kryptos genau?
Stand jetzt bieten die größten drei Währungen einen umgewandelten Wert von ?Euro und sind somit eine echte Alternative
zu klassischen Währungen geworden.
Auch die Akzeptanz der Währung bei Dienstleistern wächst immer weiter.
Und neben den konzeptuellen Erfolgen von Kryptowährungen bieten sie ebenfalls Sicherheit für Menschen weltweit, die in der
Nutzung ihrer gängigen Fiat-Währung eingeschränkt sind, sei es durch Wertmanipulation oder Unterdrückung, da das Halten
von Kryptowährungen lediglich das Merken der Seed-Phrase aus Kapitel~\ref{subsec:wallets} erfordert und somit im
Extremfall kaum nachverfolgbar ist.
Jedoch sollte man sich immer bewusst sein, dass physischer Zugriff auf Wallets oder deren Benutzer immer ein großes Problem
sein kann.
Auch sollte man nie einem System vertrauen, dass man nicht versteht oder nicht geprüft hat, was die Einstiegshürde für
Kryptowährung deutlich erhöht.
Dennoch sind sie eine vielversprechende Alternative zu klassischen Systemen und fügen sich der Idee des Web 3.0 nahtlos
an.
Die Entwicklung von Kryptowährungen sollte in den kommenden Jahren mit Spannung weiterverfolgt werden.