In Kapitel~\ref{sec:krypto} wurde gezeigt, wie eine Kryptowährung aufgebaut werden kann, um einige Probleme von klassischen
Währungen zu eliminieren.
Im Folgenden sollen noch einmal kurz die Vorteile der Nutzung einer Kryptowährung gezeigt werden, um danach genauer
auf die Nutzung einer solchen Währung einzugehen.

Grundsätzlich implementieren die meisten Kryptowährungen die grundlegenden Ideen des \emph{Web 3.0}, konkret folgende:

\begin{itemize}
    \item \textbf{Unabhängigkeit von zentralen Instanzen}:
    Kryptowährungen geben die Verantwortung der Währung den Nutzern, statt sie zentralen Instanzen anzuvertrauen und sich
    Manipulationen durch diese und Angriffen auf diese auszusetzen.
    \item \textbf{Kryptografische Absicherung}:
    Kryptowährungen basieren in keiner Form auf Vertrauen, sondern stets auf kryptografischer Absicherung und der quasi
    Unmöglichkeit, die Verfahren zu umgehen, um die Währung zu manipulieren.
    \item \textbf{Privatsphäre}:
    Kryptowährungen verzichten in der Regel auf Identifizierungen der Nutzer und verfolgen die Ansicht, dass die Nutzung
    einer Währung unter die zu schützende Privatsphäre einzuordnen ist.
    Durch den Verzicht der Speicherung von sensiblen Daten machen sich Kryptowährungen auch nicht im Sinne der Privatsphäre
    der Nutzer angreifbar.
    \item \textbf{Wertstabilität}:
    Durch eine kontrollierte Ausschüttung der Währungseinheiten und die Limitierung dieser entziehen sich Kryptowährungen
    von jeglicher Inflation, Deflation oder vergleichbaren Wertinstabilitäten.
    Sie sind somit langfristig planbar, zuverlässig und vor Wertmanipulation geschützt.
\end{itemize}

Durch diese Eigenschaften heben sich Kryptowährungen von aktuellen \emph{Fiat}-Währungen ab und bieten somit eine
Alternative.
Im Folgenden soll genauer darauf eingegangen werden, wie Kryptowährungen genutzt werden können und wie die Nutzung konkret
implementiert sein könnte.
Hierfür wird erneut die Kryptowährung \emph{Bitcoin} als Beispiel verwendet.

\subsection{Krypto-Wallets}
\label{subsec:wallets}

Wie bereits angesprochen versuchen Kryptowährungen die Identifizierbarkeit aus der Währung zu entfernen.
Primärer Fokus liegt hier bei der Eliminierung persönlicher Daten aus der eindeutigen Zuordnung des Nutzers.
Die Herausforderung liegt also darin, eine Art Konto für eine Währung anzulegen, für das man sich nicht bei einer zentralen
Bank identifizieren muss und das nicht von anderen Nutzern oder Angreifern manipuliert werden kann.

Auch hierfür bedienen sich gängige Kryptowährungen der Kryptografie, indem sie zur eindeutigen Identifizierung eines Nutzers
asymmetrische Verschlüsselungsverfahren anwenden.

\begin{figure}
    \centering
    \includegraphics[width=0.4\textwidth]{/home/nlehmann/Code/anonymitaet_in_krypto/images/keygeneration}
    \caption{Erstellung eines Key-Pairs}
    \label{fig:key}
\end{figure}

In Abbildung~\ref{fig:key} wird anschaulich dargestellt, wie Nutzer beliebig viele Schlüsselpaare erzeugen können, aus
denen dann Adressen zur Identifizierung erzeugt werden können.

\begin{enumerate}
    \item Erstellen einer \emph{Seed-Phrase} aus menschenlesbaren Wörtern
    \item Hashing der Seed-Phrase zum Erstellen des Seeds
    \item Ableitung beliebig vieler Schlüsselpaare zur Adressgenerierung und weiteren Nutzung
\end{enumerate}

Hierbei wird Schritt 1 in vielen gängigen Kryptowährungen angewandt, um das Merken des Seeds zu vereinfachen.
Ein Nutzer kann somit alle seine Schlüsselpaare regenerieren, indem er sich die Wörter der Seed-Phrase merkt.

Aus einem Schlüsselpaar kann nun eine Adresse erzeugt werden, wie Abbildung~\ref{fig:adresse} zeigt.
Diese Adresse können nun der Nutzer und alle anderen Teilnehmer nutzen, um aus der Blockchain alle Transaktionen mit
dieser Adresse auszulesen.
Durch diese Transaktionen kann dann der Kontostand des Nutzers ermittelt werden, wodurch spätere Transaktionen validiert
werden können.
Abbildung~\ref{fig:adresse} zeigt in folgenden Schritten, wie eine Bitcoin Adresse aus einem Schlüsselpaar generiert
werden kann:

\begin{figure}
    \centering
    \includegraphics[width=0.4\textwidth]{/home/nlehmann/Code/anonymitaet_in_krypto/images/adresse}
    \caption{Erstellung einer Adresse}
    \label{fig:adresse}
\end{figure}

\begin{enumerate}
    \item Doppel-Hash des Public-Keys zur Verkleinerung dessen auf einen Wertebereich von \num{160} Bit
    \item Kodierung des Hashes zu \emph{Base58}
\end{enumerate}

Der Doppel-Hash in Schritt \num{1} ist in diesem konkreten Fall eine redundante Vorsichtsmaßnahme, um die Wahrscheinlichkeit von
Kollisionen der Hashes weiter zu verringern.
Das eigentliche Ziel des Hashes ist es, den Public-Key deutlich zu verkleinern, um die Nutzung der späteren Adresse
zu vereinfachen.
Auch Schritt \num{2} dient lediglich der einfacheren Nutzung, da \emph{Base58} ähnlich scheinende Zeichen aus dem Zeichensatz entfernt.
Somit sollen Verwechslungen vermieden werden.
Im Grunde kann also der Public-Key zur Identifizierung eines Nutzers genutzt werden, in der Praxis wandelt man diesen
zur einfacheren Nutzung allerdings etwas ab.
Daraus resultieren die Wallet-Adressen.

Wie es bei asymmetrischen Verschlüsselungsverfahren gängig ist, kann nun der Public-Key des Schlüsselpaars an alle Nutzer im Netzwerk
weitergegeben werden.
Der Private-Key sollte dabei nie preisgegeben werden.

Führt man nun eine Transaktion aus, so signiert der Sender die Transaktion mit seinem Private-Key.
Somit ist garantiert, dass die Transaktion definitiv vom entsprechenden Nutzer angestoßen wurde.
Der Empfänger und alle anderen Teilnehmer im Netzwerk können diese Transaktion verifizieren, indem sie sie mit dem
Public-Key des Senders entschlüsseln.
Der Inhalt einer Transaktion besteht somit nur aus zwei Adressen - der des Senders und der des Empfängers - und dem Betrag,
der transferiert werden soll.
Somit ist die Datenmenge, die an alle Nutzer bekannt gegeben werden muss, ebenfalls auf ein Minimum beschränkt.
Abbildung~\ref{fig:transaktion} zeigt diesen Ablauf schematisch.

\begin{figure}
    \centering
    \includegraphics[width=0.4\textwidth]{/home/nlehmann/Code/anonymitaet_in_krypto/images/transaction}
    \caption{Transaktionsablauf}
    \label{fig:transaktion}
\end{figure}

\subsection{Arten von Wallets}

Wie in Kapitel~\ref{subsec:wallets} beschrieben ist eine Wallet nichts anderes als ein Werkzeug, welches für einen Nutzer
die Adressen, Schlüsselpaare und Transaktionsabläufe betreut.
Dabei limitieren sich Wallets keineswegs auf die Nutzung für nur eine Kryptowährung, stattdessen können sie beliebig komplex
ausfallen und verschiedene Kryptowährungen für einen Nutzer verwalten.
Tabelle~\ref{tab:wallets} zeigt verschiedene Implementierungsarten von Wallets für Kryptowährungen.
\begin{table}[]
    \begin{tabular}{@{}llll@{}}
        \toprule
        \multicolumn{2}{c}{\textbf{Hot Wallets}} & \multicolumn{2}{c}{\textbf{Cold Wallets}} \\ \midrule
        \multicolumn{2}{c}{Web-Anwendung} & \multicolumn{2}{c}{Hardware} \\
        \multicolumn{2}{c}{Mobile-Anwendung} & \multicolumn{2}{c}{Digital} \\
        \multicolumn{2}{c}{Desktop-Anwendung} & \multicolumn{2}{c}{Papier} \\ \midrule
        \multicolumn{1}{c}{\textcolor{green}{\textbf{+}}} &
        \begin{tabular}[c]{@{}l@{}}
            einfach\\ kostenfrei
        \end{tabular} &
        \multicolumn{1}{c}{\textcolor{green}{\textbf{+}}} &
        sehr sicher \\
        \multicolumn{1}{c}{\textcolor{red}{\textbf{--}}} & unsicher &
        \multicolumn{1}{c}{\textcolor{red}{\textbf{--}}} &
        \begin{tabular}[c]{@{}l@{}}
            teuer\\ schwerer Einstieg
        \end{tabular} \\
        \bottomrule
    \end{tabular}
    \caption{Arten von Krypto-Wallets}
    \label{tab:wallets}
\end{table}
Wallets kann man grundsätzlich in zwei Arten unterscheiden, \emph{Cold-Wallets} und \emph{Hot-Wallets}.
Man betrachtet Hot-Wallets im Allgemeinen als einfacher zu Nutzen und günstiger in der Anschaffung.
Da sie jedoch, wie der Name impliziert, immer am Netzwerk hängen, machen Hot-Wallets sich leicht angreifbar und sind
für die Nutzung mit großen Beträgen nicht zu empfehlen.

Cold-Wallets hingegen sind oft komplizierter in der Nutzung und oftmals auch teuer in der Anschaffung, sind aber deutlich
sicherer als Hot-Wallets, insofern sie korrekt genutzt werden.
Wie der Name schon impliziert, hängen Cold-Wallets nicht permanent am Netzwerk und sind somit in der Zeit, in der sie
physisch getrennt sind, nicht über das Netzwerk angreifbar.

\subsection{Arten von Kryptowährungen}

Da \emph{Bitcoin} allgemein als die erste funktionierende und tatsächlich implementierte Kryptowährung gilt unterteilt
man den Begriff Kryptowährung in der Regel in zwei Unterbereiche: \emph{Bitcoin} und \emph{Altcoin}.
Altcoin beschreibt alle Währungen außer \emph{Bitcoin} und wird selbst noch einmal in zwei Unterkategorien geteilt:\linebreak
Bitcoin\hyph{}Abwandlungen und Kryptowährungen mit nativen Blockchains.
Bitcoin\hyph{}Abwandlungen sind hierbei Währungen, welche lediglich Teile der Bitcoin Implementierung abgewandelt haben und deren
Blockchain in den frühen Blöcken identisch mit der von Bitcoin ist.
Währungen mit einer nativen oder eigenen Blockchain sind seit Beginn ihrer Blockchain unabhängig von Bitcoin entwickelt worden.
Abbildung~\ref{fig:coins} zeigt die Unterteilung von Kryptowährungen und gibt einige Beispiele für die Unterkategorien an.

\begin{figure}
    \centering
    \includegraphics[width=0.4\textwidth]{/home/nlehmann/Code/anonymitaet_in_krypto/images/waehrungen}
    \caption{Kryptowährungen}
    \label{fig:coins}
\end{figure}
